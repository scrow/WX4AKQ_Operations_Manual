\documentclass[pdflatex,letterpaper,twoside,12pt]{book}

\title             {SKYWARN Operations Manual}
\author            {Maintained by Steve Crow KG4PEQ\\Amateur Radio Coordinator}
\date              {08-Feb-2014}
\newcommand\docver {Version 2014.3}

\usepackage{skywarnbook}
\skywarnFormat{printing}
\disableAutoNumbering
\useColorLinks

\begin{document}
\skywarnTitlePage
\skipToTOC
\skywarnTOC

%%%%%%%%%%%%%%%%%%%%%%%%%%%%%%%%%%%%%%%%%%%%%%%%%%%%%%%%%%%%%%%%%%%%%%%%
%%%%%%%%%%%%%%%%%%%%%%%%%%%%%%%%%%%%%%%%%%%%%%%%%%%%%%%%%%%%%%%%%%%%%%%%
%%%%%%%%%%%%%%%%%%%%%%%%%%%%%%%%%%%%%%%%%%%%%%%%%%%%%%%%%%%%%%%%%%%%%%%%
%%%%%%%%%%%%%%%%%%%%%%%%%%%%%%%%%%%%%%%%%%%%%%%%%%%%%%%%%%%%%%%%%%%%%%%%
%%%%%%%%%%%%%%%%%%%%%%%%%%%%%%%%%%%%%%%%%%%%%%%%%%%%%%%%%%%%%%%%%%%%%%%%
%%%%%%%%%%%%%%%%%%%%%%%%%%%%%%%%%%%%%%%%%%%%%%%%%%%%%%%%%%%%%%%%%%%%%%%%

\chapter{Introduction}

%%%%%%%%%%%%%%%%%%%%%%%%%%%%%%%%%%%%%%%%%%%%%%%%%%%%%%%%%%%%%%%%%%%%%%%%

\section{Purpose of this Manual}

This manual is designed to be used as a reference guide for Wakefield SKYWARN Amateur Radio Support Team operations in the area that comprises the County Warning Area (CWA) of the National Weather Service (NWS) Weather Forecast Office (WFO) in Wakefield, Virginia.

The Wakefield NWS office and Wakefield SKYWARN Amateur Radio Support Team have a large area of responsibility which includes numerous counties and cities in portions of three states.  Some variation in local operating practices and tastes is expected and allowed, and this document serves only as a baseline reference to establish and ensure continuity of operations.  Policies and procedures may vary slightly from one Operating Area to another and from time to time.  These variances may not be immediately reflected in this Manual.

%%%%%%%%%%%%%%%%%%%%%%%%%%%%%%%%%%%%%%%%%%%%%%%%%%%%%%%%%%%%%%%%%%%%%%%%

\section{Purpose of SKYWARN}

SKYWARN is the NWS national network of trained volunteer severe storm Spotters. SKYWARN volunteers support their local community and government by providing the NWS with timely and accurate severe weather reports.  These reports, when integrated with modern NWS technology, are used to issue timely and accurate warnings of impending dangerous weather conditions.  In addition, working with emergency management officials, SKYWARN Spotters can help provide their communities with advance warning of impending hazardous weather and provide the real-time ground truth required to appropriately respond to these threats.

%%%%%%%%%%%%%%%%%%%%%%%%%%%%%%%%%%%%%%%%%%%%%%%%%%%%%%%%%%%%%%%%%%%%%%%%

\section{Role of Amateur Radio in SKYWARN}

Amateur radio has been, and probably always will be, a vital link in the NWS warning system.  Fortunately, in the Wakefield CWA there are thousands of trained SKYWARN Spotters, a large percentage of which are amateur radio operators.  Amateur radio operators possess many characteristics that make them ideal members of the SKYWARN team.  It is the desire of the National Weather Service to utilize to the fullest possible extent all the capabilities and technologies that amateur radio has to offer.

The SKYWARN Radio Desk and all equipment is either donated by public service-minded amateurs or has been purchased by the National Weather Service.  All equipment is maintained by amateur radio operators volunteering their time and expertise.

The close working relationship between the NWS and the amateur radio community provides many special benefits to each group. These benefits are highlighted in the following goals for the Wakefield SKYWARN Amateur Radio Support Team:

\begin{enumerate}
\item To provide the NWS with timely and accurate severe weather reports via amateur radio. This includes both incoming reports of severe weather per the NWS criteria and amateur radio operators making observations at specific locations in response to NWS requests.
\item To create and maintain an organized communication network for passing critical severe weather traffic in a timely fashion to and from the NWS in the event that normal communications are disrupted.  The NWS has lost normal communications in the past and it is likely the SKYWARN Radio Desk would be activated in future communication emergencies.
\item To disseminate warnings, statements, and other products issued by the NWS to the amateur radio community.  Every attempt is made to disseminate all statements and warnings issued by NWS over the SKYWARN Net to keep amateurs informed of developing situations and to practice for situations when normal communication channels fail.
\item To organize and train amateur radio operators to prepare themselves and their families for disaster or emergency weather related situations so that they make be available to assist in emergency net operations.  This preparedness training is critical if the SKYWARN system is to operate reliably during true emergency situations.
\item To maintain a Spotter network that is transparent to jurisdictional and political boundaries and operates uniformly across the entire CWA.  The Team is not directly affiliated with any club, group, or organization.
\end{enumerate}

%%%%%%%%%%%%%%%%%%%%%%%%%%%%%%%%%%%%%%%%%%%%%%%%%%%%%%%%%%%%%%%%%%%%%%%%

\section{SKYWARN's Relationship to ARRL/ARES/RACES}

Amateur radio's participation in the NWS SKYWARN program is formally acknowledged and encouraged by a Memorandum of Understanding (MOU) between the American Radio Relay League (ARRL) and the NWS.  This agreement states that the ARRL will encourage its local volunteer groups operating as Amateur Radio Emergency Services (ARES) to provide the NWS with Spotters and communicators as needed by the NWS during severe weather.

The Radio Amateur Civil Emergency Services (RACES) is a government supported organization made up of amateur radio operators registered with state, county, or local offices of emergency management. In Virginia, these amateurs are also ARES members.

Many natural disasters in the Wakefield CWA are the direct result of severe weather and/or are exacerbated by severe weather.  Accordingly, the NWS may utilize SKYWARN amateur radio operators not only to obtain and disseminate severe weather observations and warnings, but also maintain close coordination with emergency management agencies throughout the CWA.

A good relationship between SKYWARN and these other agencies is critical to ensure the effective flow of communication and efficient utilization of available resources during emergencies.

The Wakefield SKYWARN Amateur Radio Support Team will coordinate closely with, and seek cooperation from, all ARES and RACES groups in the area.  The Team and its nets will not, however, bear the name of, or be directly tied to, ARES or RACES.  The Team and its nets are independent, open to all qualified amateur radio operators in and around the Wakefield County Warning Area.

The relationship between SKYWARN and agencies other than the NWS is generally limited to coordination of on-air activities and sharing of airspace in the event of multiple simultaneous activations.  SKYWARN will also provide a communication path between these agencies and the NWS in the event of a failure of normal communications.  SKYWARN is generally not in the business of relaying individual reports to outside agencies; instead, the NWS uses reports collected to produce official report products for dissemination to these agencies and the public.

%%%%%%%%%%%%%%%%%%%%%%%%%%%%%%%%%%%%%%%%%%%%%%%%%%%%%%%%%%%%%%%%%%%%%%%%

\section{SKYWARN and Storm Chasers}

Local perspectives on storm chasing continue to evolve with time.  As the National Weather Service finds increasing value in the contributions of qualified storm chasers outside of ``Tornado Alley'' the SKYWARN amateur radio network must evolve to support those chasers.

Storm chasing remains a secondary focus at best, and we continue to maintain a position in line with the Wakefield Forecast Office.  That is, we do not recommend, endorse, or encourage storm chasing.  However, we will communicate with, and provide reporting channels for, qualified storm chasers who wish to participate in our nets.

For purposes of this program, the term ``storm chasing'' shall be applied to the intentional pursuit of severe weather, regardless of whether such activities are for purposes of severe weather reporting, photography, videography, research, or thrill.

The National Weather Service and the NWS Wakefield SKYWARN Amateur Radio Support Team consider the safety of all SKYWARN volunteers to be of paramount importance.

SKYWARN Spotters and other persons who choose to engage in storm chasing do so at their own risk and completely independent of their involvement in the SKYWARN program.  The National Weather Service and the SKYWARN program accept no responsibility for the decisions of SKYWARN program participants with regard to storm chasing and can assume no liability for damages arising from the use of SKYWARN reports in storm chasing or any other activities.

%%%%%%%%%%%%%%%%%%%%%%%%%%%%%%%%%%%%%%%%%%%%%%%%%%%%%%%%%%%%%%%%%%%%%%%%
%%%%%%%%%%%%%%%%%%%%%%%%%%%%%%%%%%%%%%%%%%%%%%%%%%%%%%%%%%%%%%%%%%%%%%%%
%%%%%%%%%%%%%%%%%%%%%%%%%%%%%%%%%%%%%%%%%%%%%%%%%%%%%%%%%%%%%%%%%%%%%%%%
%%%%%%%%%%%%%%%%%%%%%%%%%%%%%%%%%%%%%%%%%%%%%%%%%%%%%%%%%%%%%%%%%%%%%%%%
%%%%%%%%%%%%%%%%%%%%%%%%%%%%%%%%%%%%%%%%%%%%%%%%%%%%%%%%%%%%%%%%%%%%%%%%
%%%%%%%%%%%%%%%%%%%%%%%%%%%%%%%%%%%%%%%%%%%%%%%%%%%%%%%%%%%%%%%%%%%%%%%%

\chapter{Wakefield SKYWARN Team Structure}

%%%%%%%%%%%%%%%%%%%%%%%%%%%%%%%%%%%%%%%%%%%%%%%%%%%%%%%%%%%%%%%%%%%%%%%%

\section{General Information}

SKYWARN is not a club, but rather a public service organization dedicated to service to the National Weather Service during periods of severe weather. SKYWARN is open to all qualifying amateur radio operators and is an independent group not directly affiliated with any other club, group, or organization.

SKYWARN consists of several key elements.  The first of these elements is the NWS SKYWARN Program Manager and Focal Point(s).  These individuals are NWS employees who are responsible for overseeing the operation of the network, for selecting and appointing key SKYWARN personnel, and for acting as contact points for the NWS among the amateur radio community.  The Warning Coordination Meteorologist (WCM) serves as the Program Manager, and may appoint SKYWARN Focal Point(s) to perform day-to-day functions associated with the SKYWARN network.

The SKYWARN Amateur Radio Coordinator, along with his Area Managers organize the operation of the entire SKYWARN amateur radio network in accordance with the needs and guidance set forth by the NWS.  This volunteer position is appointed by the NWS SKYWARN Program Manager to ensure that the person chosen can work well with NWS personnel and to ensure stability in the position.  The Coordinator must possess superior communication and coordination skills and should be readily available to the NWS, both for severe weather net activations and for consultation on amateur radio issues.  The \nameref{arc-jobdesc} can be found on page \pageref{arc-jobdesc} in this manual.

Nothing could be accomplished without Net Control Operators (NCO) and Responders.  Net Controllers and Responders are both responsible for operating the local SKYWARN nets during activations, and Responders have the additional role of operating the SKYWARN Radio Desk at the Wakefield WFO during particularly severe weather events or communication disruptions.

%%%%%%%%%%%%%%%%%%%%%%%%%%%%%%%%%%%%%%%%%%%%%%%%%%%%%%%%%%%%%%%%%%%%%%%%

\section{Organization Name}

The organization was originally known as Central \& Eastern Virginia SKYWARN and maintains this identity for legal identification purposes.  In the course of routine operations, it is known as the Wakefield SKYWARN Amateur Radio Support Team, or simply, Wakefield SKYWARN.

%%%%%%%%%%%%%%%%%%%%%%%%%%%%%%%%%%%%%%%%%%%%%%%%%%%%%%%%%%%%%%%%%%%%%%%%

\section{Mission Statement}

The Wakefield SKYWARN Amateur Radio Support Team exists to provide communication services for the collection of severe weather reports and dissemination of critical weather information in support of the National Weather Service and its mission to protect life and property by improving warning accuracy.

%%%%%%%%%%%%%%%%%%%%%%%%%%%%%%%%%%%%%%%%%%%%%%%%%%%%%%%%%%%%%%%%%%%%%%%%

\section{Core Values}

Since 2008 the Wakefield SKYWARN Amateur Radio Support Team has operated under seven Core Values which guide everything we do:

\textbf{Dedication.} We serve the National Weather Service and each other with a spirit of commitment and dedication to our common mission to protect life and property.

\textbf{Education.} We value each others' interests, skills, and experiences and we actively and publicly share our talents and our knowledge.

\textbf{Integrity.} We act honestly and in the best interest of SKYWARN in everything we do.

\textbf{Respect.} We recognize the duties each person on the team has volunteered to perform and we appreciate their hard work, even when things go wrong.

\textbf{Teamwork.} No one person can carry the weight of the SKYWARN program. It takes many people working together with diverse skills and a common goal to achieve success.

\textbf{Community Involvement.} We share our educational resources and our intellectual assets freely with the communities we serve and strive to be good neighbors and partners in the amateur radio world.

\textbf{Continuous Improvement.} We aim to do a lot of things right and want to be the best, but in order to truly be the best we need to accept that sometimes we'll fail, and that's okay as long as we learn something from it and make ourselves stronger.

%%%%%%%%%%%%%%%%%%%%%%%%%%%%%%%%%%%%%%%%%%%%%%%%%%%%%%%%%%%%%%%%%%%%%%%%

\section{Legal Structure}

The team currently operates without formal legal structure.  Each member is volunteering his or her time, resources, and efforts to the National Weather Service directly.  While there is some formal leadership structure to the team, it is not its own distinct legal entity.

%%%%%%%%%%%%%%%%%%%%%%%%%%%%%%%%%%%%%%%%%%%%%%%%%%%%%%%%%%%%%%%%%%%%%%%%

\section{Financial Structure}

The team does not have or manage its own finances.  Occasional funding for equipment needs is accomplished through the normal budget and procurement process at the National Weather Service, though the primary source of equipment and supplies remains individual donations from team members and partners.  The team does not maintain its own physical assets; radio equipment and operational supplies are considered property of the National Weather Service.

The giving of cash or cash-equivalent donations to the team is formally discouraged.  Individuals and organizations wishing to donate physical goods such as radio equipment, accessories, or supplies should do so to the National Weather Service.  This may be accomplished through the team's leadership structure or by direct contact with the SKYWARN Program Manager.

Lacking any financial structure or assets, the SKYWARN Amateur Radio Support Team is unable to reimburse its leadership, members, or other individuals for any equipment, supplies, services, or other purchases, or for mileage, insurance, or any other expenses.

%%%%%%%%%%%%%%%%%%%%%%%%%%%%%%%%%%%%%%%%%%%%%%%%%%%%%%%%%%%%%%%%%%%%%%%%

\section{Assets and Liabilities}

In general, all equipment installed at NWS Wakefield is the property of the United States Department of Commerce.  Certain incidental supplies, such as binders, notebooks, pens, paper, food, beverages, etc. at the SKYWARN amateur radio station are supplied by and at the sole expense of the purchasing Amateur Radio Coordinator, Area Manager, Responder, Net Control Operator, or other individual, and become the property of the SKYWARN Amateur Radio Support Team. 

%%%%%%%%%%%%%%%%%%%%%%%%%%%%%%%%%%%%%%%%%%%%%%%%%%%%%%%%%%%%%%%%%%%%%%%%

\section{Ability to Further Organize}

Nothing in this manual shall be construed as to restrict the ability to further organize the team.  There would be substantial benefit into incorporating the team as a non-profit organization, and the SKYWARN Amateur Radio Coordinator has the ability to restructure the team accordingly.

%%%%%%%%%%%%%%%%%%%%%%%%%%%%%%%%%%%%%%%%%%%%%%%%%%%%%%%%%%%%%%%%%%%%%%%%
%%%%%%%%%%%%%%%%%%%%%%%%%%%%%%%%%%%%%%%%%%%%%%%%%%%%%%%%%%%%%%%%%%%%%%%%
%%%%%%%%%%%%%%%%%%%%%%%%%%%%%%%%%%%%%%%%%%%%%%%%%%%%%%%%%%%%%%%%%%%%%%%%
%%%%%%%%%%%%%%%%%%%%%%%%%%%%%%%%%%%%%%%%%%%%%%%%%%%%%%%%%%%%%%%%%%%%%%%%
%%%%%%%%%%%%%%%%%%%%%%%%%%%%%%%%%%%%%%%%%%%%%%%%%%%%%%%%%%%%%%%%%%%%%%%%
%%%%%%%%%%%%%%%%%%%%%%%%%%%%%%%%%%%%%%%%%%%%%%%%%%%%%%%%%%%%%%%%%%%%%%%%

\chapter{SKYWARN Operating Areas}

To better organize the structure of the SKYWARN program and to ensure efficient operation and adequate network staffing during SKYWARN activations, the NWS Wakefield County Warning Area has been broken down into several Operating Areas.  The boundaries of these Operating Areas are generally based on repeater coverage.

Each Operating Area is directed by an Area Manager.  In the event of a vacancy in that position, the personnel within that Operating Area will report directly to the Amateur Radio Coordinator until the position is filled.  The basic duties of the Area Manager are outlined in this manual. 

The Area Manager oversees his group of qualified, trained Net Control Stations and ensures that at least two Net Control Operators (or the Area Manager) are available to start a net at all times, 24 hours a day, throughout the year. 

%%%%%%%%%%%%%%%%%%%%%%%%%%%%%%%%%%%%%%%%%%%%%%%%%%%%%%%%%%%%%%%%%%%%%%%%

\section{Cooperation Among Operating Operating Areas}

It is often possible for a Net Control Operator in one Operating Area to access the designated SKYWARN repeater in a neighboring Operating Area.  Area Managers and their Net Control Operators are expected to cooperate with one another by taking whatever actions are necessary to ensure neighboring nets are covered with a qualified Net Control Operator as needed. 

Net Control Operator volunteers shall provide a list of designated SKYWARN repeaters which he/she can access from the location(s) from which they will be serving as NCO, and shall provide updates to this list as station or repeater changes modify station capabilities. 

Area Managers whose territories lie at the outer boundaries of the Wakefield CWA are strongly encouraged to become familiar with and forge a positive, cooperative relationship with neighboring SKYWARN amateur radio teams.

%%%%%%%%%%%%%%%%%%%%%%%%%%%%%%%%%%%%%%%%%%%%%%%%%%%%%%%%%%%%%%%%%%%%%%%%

\section{SKYWARN Repeaters}

In the spirit of good amateur radio practice, the SKYWARN leadership will secure permission from the trustee of each Primary repeater prior to its designation as a Primary SKYWARN repeater.  Generally, no such permission will be sought for repeaters periodically used as Backup repeaters. 

Repeaters selected for Primary use should be wide-coverage repeaters readily accessible from handheld and mobile stations within the majority of the Operating Area the repeater will serve.  The repeaters should have a track record of reliability and availability, and should be equipped with a minimum four hours of battery and/or generator backup power.  Additional consideration will be given to repeaters with any sort of linking capability.

%%%%%%%%%%%%%%%%%%%%%%%%%%%%%%%%%%%%%%%%%%%%%%%%%%%%%%%%%%%%%%%%%%%%%%%%

\section{Use of Special Repeater Functions}

Many repeaters are equipped with features such as IRLP, Echolink, autopatch, telemetry functions, frequency agile remote bases, etc.  While some of these features may be available to the general public, many times these features are restricted to club or repeater members. 

A Net Control Operator who is an authorized user of these repeater features may use them in conjunction with SKYWARN activities with the approval of the repeater Trustee. 

In some instances SKYWARN has received permission to use these features directly.  The access codes for these features must not be shared with anyone, even within the SKYWARN program.  The individuals who are authorized to use these features in conjunction with their SKYWARN activities have an obligation to secure and preserve the integrity of the access codes. 

Responders operating under the WX4AKQ call sign may only use the repeater access codes for which WX4AKQ is authorized.  If a Responder has a specific authorization to use a repeater access code for SKYWARN purposes, they should identify with their own call sign while controlling the repeater.

%%%%%%%%%%%%%%%%%%%%%%%%%%%%%%%%%%%%%%%%%%%%%%%%%%%%%%%%%%%%%%%%%%%%%%%%

\section{Subnets}

Some Operating Areas may have Subnets in locations where the Primary SKYWARN Repeater does not have adequate coverage.  It is suggested, but not required, that each Subnet be under the direction of a designated Net Manager. The Net Manager role is described in this manual. 

Staffing of the Primary Net takes priority;  Net Control Operators should be assigned to the Primary Net first, and any remaining Net Control Operators may then be assigned to the Subnet.  An exception exists for situations where severe weather is confined to the territory served by the Subnet, in which case it is only necessary to activate the Subnet for that event.

%%%%%%%%%%%%%%%%%%%%%%%%%%%%%%%%%%%%%%%%%%%%%%%%%%%%%%%%%%%%%%%%%%%%%%%%

\section{Special Provisions for New Operating Areas}

To facilitate the launch of new Operating Areas or the relaunch of a defunct Operating Area, the Amateur Radio Coordinator has the right to waive some of the training, experience, and licensing requirements.  Frequency designations may be made on a temporary basis pending formal approval of repeater trustees and an evaluation of the team's capabilities.  Spotter training requirements may be relaxed provided team members are willing to obtain the required training within one year.

%%%%%%%%%%%%%%%%%%%%%%%%%%%%%%%%%%%%%%%%%%%%%%%%%%%%%%%%%%%%%%%%%%%%%%%%
%%%%%%%%%%%%%%%%%%%%%%%%%%%%%%%%%%%%%%%%%%%%%%%%%%%%%%%%%%%%%%%%%%%%%%%%
%%%%%%%%%%%%%%%%%%%%%%%%%%%%%%%%%%%%%%%%%%%%%%%%%%%%%%%%%%%%%%%%%%%%%%%%
%%%%%%%%%%%%%%%%%%%%%%%%%%%%%%%%%%%%%%%%%%%%%%%%%%%%%%%%%%%%%%%%%%%%%%%%
%%%%%%%%%%%%%%%%%%%%%%%%%%%%%%%%%%%%%%%%%%%%%%%%%%%%%%%%%%%%%%%%%%%%%%%%
%%%%%%%%%%%%%%%%%%%%%%%%%%%%%%%%%%%%%%%%%%%%%%%%%%%%%%%%%%%%%%%%%%%%%%%%

\chapter{SKYWARN Roles}

%%%%%%%%%%%%%%%%%%%%%%%%%%%%%%%%%%%%%%%%%%%%%%%%%%%%%%%%%%%%%%%%%%%%%%%%

\section{SKYWARN Amateur Radio Coordinator}

\subsection{Amateur Radio Coordinator Job Description}\label{arc-jobdesc}

The Amateur Radio Coordinator is responsible for all aspects of the operation of the amateur radio team and its relationship between the team and both the National Weather Service and various outside organizations.

The coordinator should be a trained SKYWARN amateur radio volunteer.  He/she is expected to provide technical expertise and assistance to the SKYWARN Program Manager.  The coordinator is expected to effectively coordinate SKYWARN efforts over a multi-state area with a wide variety of radio clubs, groups and organizations.  He/she should be easily accessible at all times and be willing to devote time and effort to this important cause.  The coordinator should, ideally, live within reasonable commuting distance of the NWS WFO Wakefield.  This person should have an interest in the SKYWARN program and be able to work well with the NWS SKYWARN Program Manager and other NWS staff. 

\subsection{Amateur Radio Coordinator Position Qualifications}

To be eligible to serve as the Amateur Radio Coordinator, the candidate must meet the following criteria:

\begin{enumerate}
\item Must meet all \nameref{nco-criteria} as specified on page \pageref{nco-criteria} of this manual.
\item Must possess a valid Extra Class amateur radio license.
\item Must possess and maintain valid Basic SKYWARN Spotter training from the Wakefield WFO and be willing to obtain Advanced SKYWARN Spotter certification within one year.
\item Must have proof of completion of current FEMA IS.100, IS.200, IS.700, and IS.800 ICS/NIMS training.
\item Must be willing and able to respond to the WFO for SKYWARN activations and other administrative tasks as needed.
\item Must be willing and able to attend Spotter training events and provide outreach services across the entire CWA throughout the year.
\item Must have strong communication and leadership skills.  A professional background in a supervisory or management role is preferred.
\item Must not be serving in a leadership or supervisory role with any other emergency communications organization (for example, ARES/RACES, Red Cross, VDEM, etc).
\item Must have reliable access to the Internet and all potential frequencies and modes of communication (namely VHF FM, HF SSB, APRS).
\end{enumerate}

Additional qualifying conditions may be imposed by the SKYWARN Program Manager.

\subsection{Selection of the Amateur Radio Coordinator}

As the served agency, the National Weather Service will be responsible for the selection and appointment of the Coordinator, and any Area Managers).  The Coordinator may serve as long as his service is in the best interest of SKYWARN.  In the event that the Coordinator position is left vacant, the NWS Program Manager may ask the remaining members of the SKYWARN Leadership Team to assist with suggestions or nominations for the position.

\subsection{Duties of the Amateur Radio Coordinator}

The basic duties of the SKYWARN Amateur Radio Coordinator are as follows.

\begin{enumerate}
\item Serve as a SKYWARN Spotter and attend training classes whenever possible.  Attendance of Spotter training classes is intended both to solidify personal training as a Spotter and also to provide representation for the amateur radio team.  The Amateur Radio Coordinator should take a few minutes during each class to speak to the amateur radio operators and prospective hams in the room about the SKYWARN Amateur Radio Support Team, including amateur radio's role in SKYWARN and how to get involved personally.
\item Work with Area Managers in the organization and operation of the SKYWARN network, including the recruitment and training of Responders, Net Control Operators, and Spotters.
\item Assist the NWS SKYWARN Program Manager as a technical expert on amateur radio matters.
\item Ensure that at least five Area Managers and/or Responders are on call at all times to receive the activation call from the National Weather Service.
\item Ensure that at least two Responders are dispatched to the National Weather Service when needed or requested by NWS staff.
\item Assist in creation and maintenance of the SKYWARN Operations Manual.
\item Coordinate simple and effective procedures for communication between the WFO and the local SKYWARN nets through the use of the most effective frequencies and modes.
\item Coordinate SKYWARN activities with ARES/RACES and other amateur radio and government groups and agencies to best fulfill the goals of the NWS SKYWARN program.
\item Maintain the SKYWARN Radio Desk and assist in the acquisition of new equipment when necessary.  See \nameref{wx4akq-maint} on page \pageref{wx4akq-maint}.
\item Coordinate with SKYWARN nets in adjacent CWA's to establish backup procedures and to share ideas and strategies.
\item Coordinate, attend, and participate in SKYWARN Leadership meetings/calls.
\item Issue SKYWARN Risk Assessment Bulletins or other communications to keep all team members abreast of important weather conditions which may impact the team throughout the next 24 to 36 hours.  These bulletins are covered in ``\nameref{risk-assessments}'' on page \pageref{risk-assessments}.
\end{enumerate}

%%%%%%%%%%%%%%%%%%%%%%%%%%%%%%%%%%%%%%%%%%%%%%%%%%%%%%%%%%%%%%%%%%%%%%%%

\section{Area Managers}

\subsection{Area Manager Job Description}\label{am-jobdesc}

The Area Manager is responsible for all local SKYWARN operations within his or her assigned Operating Area.  From time to time, an Area Manager may be asked to help with activities in another Operating Area or at the WFO, and an Area Manager may be assigned to more than one Operating Area. 

\subsection{Area Manager Position Qualifications}

To be eligible to serve as Area Manager, the following basic conditions must be met by the candidate:

\begin{enumerate}
\item Must meet all \nameref{nco-criteria} as specified on page \pageref{nco-criteria} of this manual
\item Must possess and maintain valid Basic SKYWARN Spotter training from the Wakefield WFO and be willing to obtain Advanced SKYWARN Spotter certification within one year.
\item Must be willing and able to periodically travel to the Wakefield WFO for administrative tasks as needed.
\item Must be willing and able to attend Spotter training events and provide outreach services across the assigned Operating Area throughout the year.
\item Must have strong communication and leadership skills.  A professional background in a supervisory or management role is preferred.
\item Must not be serving in a leadership or supervisory role with any other emergency communications organization (for example, ARES/RACES, Red Cross, VDEM, etc) where such involvement may adversely impact the ability to adequately serve SKYWARN during activations.  Exceptions may be made for short-term/interim appointments expected to last 6 months or less.
\item Must have reliable access to the Internet and all potential frequencies and modes of communication (namely VHF FM, HF SSB, APRS).
\end{enumerate}

\subsection{Selection of the Area Manager}

The SKYWARN Amateur Radio Coordinator may appoint, in consultation with the SKYWARN Program Manager, one or more Area Managers to assist him in his role as Coordinator.  It is generally preferred to maintain a ratio of one Area Manager per Operating Area except in the case of newly formed Operating Areas or the promotion of a new Area Manager, in which case an additional, seasoned Area Manager may be assigned. 

\subsection{Duties of the Area Manager}

The Area Manager's duties are similar to those of the Coordinator, but the focus is more local.  In the Wakefield SKYWARN program, each Area Manager will generally be responsible for a single Operating Area and will focus on the recruiting, training, and activation of qualified Net Control Operators for his individual territory. 

The Area Manager is expected to attend as many local NWS SKYWARN Spotter training classes as possible and engage in other community service, public relations, or program awareness activities at the local level.  This may include periodic SKYWARN outreach presentations to local clubs.

Additionally, forming and maintaining operating agreements with local repeater operators within each Operating Area are the responsibility of the Area Manager.

The duties of the Amateur Radio Coordinator may periodically be delegated to one or more Area Managers.

\subsection{Assistant Area Manager}

An Area Manager may appoint one or more Assistant Area Managers to share some administrative responsibility.  Ultimately, the Area Manager is fully responsible for the performance of his or her assigned area(s).

%%%%%%%%%%%%%%%%%%%%%%%%%%%%%%%%%%%%%%%%%%%%%%%%%%%%%%%%%%%%%%%%%%%%%%%%

\section{Net Managers}

\subsection{Net Manager Job Description}\label{nm-jobdesc}

A Net Manager may be appointed to oversee a local Subnet.  The Net Manager is responsible for ensuring the Subnet activates and operates as needed for severe weather events and provides coordination of Net Control Operator resources between the Subnet and the Primary Net.

\subsection{HF Net Manager}

An HF Net Manager shall be appointed specifically for purposes of maintaining a roster of Net Control Operators with HF communication capabilities, and shall be responsible for ensuring the HF net is operational during times of severe weather and during other activations as required.

All HF Operating Areas are under the direction of one or more Net Managers, who report directly to the Amateur Radio Coordinator.  The scope of responsibility of HF Net Managers is the same as that of any other Net Manager.  Outreach services, training, and most administrative duties are primarily the responsibility of the Amateur Radio Coordinator, though various tasks may be delegated to the Net Manager from time to time.

\subsection{Net Manager Position Qualifications}

The Net Manager shall meet all of the \nameref{nco-criteria} specified on page \pageref{nco-criteria}.  Additionally, the Net Manager shall possess the communication capabilities required for the operation of his designated net, and should have the basic leadership capabilities required to properly manage the Net Control Station resources assigned to his net.

\subsection{Selection of Net Managers}

Area Managers are responsible for selection of Net Managers serving a Subnet within their assigned Operating Area(s) and may freely add or remove Net Managers as desired.  HF Net Managers are appointed by the Amateur Radio Coordinator.  The Amateur Radio Coordinator may also directly add or remove Net Managers for a local Subnet as needed.

\subsection{Duties of the Net Manager}

The Net Manager shall:

\begin{enumerate}
\item Use existing Net Control Operators to staff his Subnet after the Primary Net is properly staffed.
\item Coordinate the relay of reports from the Subnet to either the Primary Net or the National Weather Service as may be necessary during a communications or systems outage.
\item Work with the Area Manager to address staffing deficiencies and other operational challenges.
\item The HF Net Manager reports directly to the Amateur Radio Coordinator instead of an Area Manager.
\end{enumerate}

%%%%%%%%%%%%%%%%%%%%%%%%%%%%%%%%%%%%%%%%%%%%%%%%%%%%%%%%%%%%%%%%%%%%%%%%

\section{Net Control Operator}

\subsection{Net Control Operator Job Description}

The Net Control Operator (NCO) is the most critical position in any SKYWARN activation.  It is a role that challenges all of an amateur radio operator's communications and technical skills.  It is also an extremely responsible position in that the safety of lives and property may rest on the amateur's skills.  Although this role is challenging, with proper training and experience, it can also be extremely rewarding when a net is run effectively.

The Reserve Net Control Operator (Reserve NCO) position is available to persons interested in serving as a SKYWARN Net Control Operator on a less frequent basis.  Individuals with availability restrictions or other commitments which limit their ability to serve SKYWARN may join as a Reserve NCO.  Additionally, regular NCO's who have not participated in SKYWARN activations for an extended period of time may be re-classified as a Reserve NCO.

\subsection{Net Control Operator Position Qualifications}\label{nco-criteria}

To serve as a Net Control Operator the following criteria must be met:

\begin{enumerate}
\item Must possess a valid Technician Class or higher amateur radio license.  A General Class license or higher is preferred, especially in situations where the Net Control Operator may be required to communicate with WX4AKQ or other EMCOMM entities via HF.
\item Must possess and maintain valid Basic or Advanced SKYWARN Spotter training from the Wakefield WFO, or obtain this training within 90 days of joining the team.
\item Must attend and complete a SKYWARN Net Control Operator training course prior to enlisting as a member and must attend refresher training at least once every three years or as otherwise directed.  See \nameref{nco-training} on page \pageref{nco-training}.
\item Must have a computer and Internet access for logging and relaying of reports.  Short-term backup power for the computer and Internet connectivity is preferred.  Must be reasonably comfortable with the operation of the computer, including web browser, basic productivity software, and e-mail.
\item Must possess and maintain sufficient communications equipment to be able to reliably operate the primary SKYWARN net (see Station Requirements, below).
\item Must have excellent organization and communications skills.  Past experience as a Net Control for any EMCOMM organization is preferred but not required.
\item Must have a cell phone or pager for notification purposes;  mobile ``push'' e-mail capability is highly recommended.
\item Must have a working NOAA Weather Radio receiver;  SAME capability is recommended.
\end{enumerate}

All of the preceeding criteria also apply to Reserve Net Control Operators, except:

\begin{enumerate}
\item Must attend and complete a SKYWARN Net Control Operator training annually unless minimum participation expectations are met.
\end{enumerate}

If a Reserve NCO logs at least one report during a SKYWARN activation in the course of a calendar year, the mandatory 1-year re-certification requirement will usually be waived.  The Reserve NCO may opt to take the 1-year re-certification training, and is encouraged to do so, but it will not usually be required.  In any event, the Area Manager may choose to require more frequent training of any NCO, including a Reserve NCO, based on performance or other needs.

\subsection{Net Control Operator Station Requirements}\label{nco-station}

It is essential that the Net Control Operator's station meet minimum design and performance standards to ensure reliability and continued service in adverse operating conditions.

\begin{enumerate}
\item The station must be able to reliably access the primary SKYWARN repeater and at least one backup repeater within the Operating Area.  In most areas, this requires only a 2-meter radio.
\item The station must be equipped with backup power capable of sustaining continuous net operation for a period of at least three hours.  This backup power may be any combination of battery, generator, or alternative energy source.
\end{enumerate}

Local amateur radio clubs can prove an invaluable tool in designing and implementing a proper backup power system.  Prospective Net Control Operators are encouraged to reach out to other amateurs for assistance with their backup power system if building one from scratch to support SKYWARN.

Reserve NCO station requirements are the same, except:

\begin{itemize}
\item Backup power requirements are waived.
\end{itemize}

All other regular NCO station requirements remain in place for Reserve NCO's.

\subsection{Selection of Net Control Operators}

Net Control Operators are recruited from among the general amateur radio population and are not specifically selected or appointed by SKYWARN leadership.  Any interested amateur who qualifies and completes the required training may participate as a Net Control Operator.

\subsection{Duties of the Net Control Operator}\label{nco-duties}

The Net Control Operator is responsible for:

\begin{enumerate}
\item Operating the SKYWARN Net in accordance with prescribed operating procedure.
\item Managing net check-ins and check-outs and maintaining a list of stations currently checked in.
\item Appropriately managing the flow of traffic on the repeater during both informal and directed nets.
\item Exercising good judgment in the transition to and from a directed net as required based on traffic flow and current warnings.
\item Logging all reports received into the net, regardless of whether the report meet \nameref{reporting-criteria}.
\item Relaying those reports which meet \nameref{reporting-criteria} to the National Weather Service.
\item Executing requests for reports from specific areas as may be received from NWS employees, Area Managers, or other channels.
\item Constructively and creatively working alongside other nets which may be active on the same frequency at the same time during large-scale or high-impact events, for example, ARES nets.
\item Ensuring the frequency remains clear for regular amateur radio use during informal nets and that the frequency remains clear for SKYWARN traffic during directed nets.
\item Disseminating new watches, warnings, advisories, and statements over the air and periodically providing a recap of existing products.
\item Handling other incidental emergency traffic as may occur from time to time.
\end{enumerate}

Net Control Operators are truly on the front lines of the SKYWARN program and are expected to conduct their nets in a professional and courteous manner at all times.

\subsection{Activity Evaluation and Classification}

Net Control Operators are expected to regularly volunteer to serve as Net Control for SKYWARN activations.  This includes periodically operating as a primary Net Control, logging reports from Spotters.  Net Control Operator activity will be evaluated based on the number of reports logged and number of days active in the previous 18 months.  Net Control Operators who do not log at least one report in a rolling 18-month period will be systematically re-classified as a Reserve NCO and will be subject to the training requirements applicable to that position.

Reserve NCO's wishing to be moved to regular NCO status can request this change through their Area Manager.

\subsection{Net Call-Up Preference}

Net Control Operators should proactively volunteer for NCO duty when available.  If there is a shortage of volunteers, the Area Manager will begin contacting individual NCO's via telephone, e-mail, or radio.  Regular Net Control Operators will be called first.  If there are insufficient NCO's at that point, the \nameref{rnco} list will be used.

\subsection{Mix of Regular vs. Reserve NCO's}

Area Managers should maintain a minimum of 3 regular NCO's and 3 Reserve NCO's to ensure adequate staffing levels are available.  Aside from the minimum headcount, there is no required ratio of regular to Reserve NCO's.

%%%%%%%%%%%%%%%%%%%%%%%%%%%%%%%%%%%%%%%%%%%%%%%%%%%%%%%%%%%%%%%%%%%%%%%%

\section{Responder}

\subsection{Position Description}

The Responder is a specially-trained operator who staffs the SKYWARN Radio Desk inside the Wakefield WFO.

\subsection{Position Qualifications}

To serve as a Responder the following criteria must be met:

\begin{enumerate}
\item Must meet all \nameref{nco-criteria} as specified on page \pageref{nco-criteria} of this manual.
\item Must possess a valid General Class or higher amateur radio license.
\item Must possess and maintain valid Basic SKYWARN Spotter training from the Wakefield WFO.
\item Must be willing to travel to the WFO for SKYWARN activations, which may occur on very short notice and may run anywhere from a few hours to a couple of days.
\item Must possess the skills necessary to work with a number of Net Control Operators, Area Managers, and NWS employees.
\item Must be familiar with standard VHF and HF net operating procedures.
\item Must be familiar with APRS and Winlink messaging functions, including the XASTIR software.
\item Must have proof of completion of FEMA IS.100, IS.200, IS.700, and IS.800 training, or be willing and able to complete this training within 90 days of appointment to the Responder role.
\end{enumerate}

\subsection{Selection of Responders}

SKYWARN Responders are selected by the SKYWARN Amateur Radio Coordinator and the Area Managers.  Responders are selected on the basis of their availability to operate the NWS Radio Desk and their individual experience and other qualifications.  Any Net Control Operator who is interested in becoming a Responder should contact the Amateur Radio Coordinator.

\subsection{Duties of the Responder}

The Responder is responsible for:

\begin{enumerate}
\item Being readily available to report to the WFO on short notice within his or her specified availability.
\item Effectively moving reports and radio traffic between the individual SKYWARN nets and the WFO.
\item Exercising good judgment in the utilization of repeater linking technologies and remote bases which the SKYWARN program has received specific authorization to use, and vigorously defending the confidentiality of the control codes associated with these systems.
\item Serving as an on-air interface between the WFO and various emergency service organizations which may be active during large-scale and high-impact weather events.
\item Periodically running individual SKYWARN nets in the absence of a Net Control Operator.
\item Efficiently working alongside the NWS employees at the WFO in the movement of reports and requests for reports between the WFO and SKYWARN nets.
\end{enumerate}

To the NWS employees in the WFO, Responders are the ``faces'' of amateur radio.  All Responders are required to maintain a professional appearance and attitude, treat the NWS employees and other SKYWARN volunteers with respect and courtesy, and continuously maintain a high standard of conduct when present in the WFO. Any misconduct on the part of a Responder may result in immediate discharge from the WFO and the SKYWARN program.

\subsection{Substitution with Net Controllers and Area Managers}

In extreme circumstances when no Responder is available to operate the SKYWARN Radio Desk, one or more Net Control Operators and/or Area Managers who otherwise possess the necessary license class and experience may be dispatched to the WFO to serve in the Responder role.

%%%%%%%%%%%%%%%%%%%%%%%%%%%%%%%%%%%%%%%%%%%%%%%%%%%%%%%%%%%%%%%%%%%%%%%%
%%%%%%%%%%%%%%%%%%%%%%%%%%%%%%%%%%%%%%%%%%%%%%%%%%%%%%%%%%%%%%%%%%%%%%%%
%%%%%%%%%%%%%%%%%%%%%%%%%%%%%%%%%%%%%%%%%%%%%%%%%%%%%%%%%%%%%%%%%%%%%%%%
%%%%%%%%%%%%%%%%%%%%%%%%%%%%%%%%%%%%%%%%%%%%%%%%%%%%%%%%%%%%%%%%%%%%%%%%
%%%%%%%%%%%%%%%%%%%%%%%%%%%%%%%%%%%%%%%%%%%%%%%%%%%%%%%%%%%%%%%%%%%%%%%%
%%%%%%%%%%%%%%%%%%%%%%%%%%%%%%%%%%%%%%%%%%%%%%%%%%%%%%%%%%%%%%%%%%%%%%%%

\chapter{Recruiting and Outreach}

The continued success of the SKYWARN team depends on a non-stop evolutionary cycle of teaching, developing, and advancing current members and bringing new members into the program.  Outreach functions involve establishing and maintaining relationships with our partners in the communities we serve.

%%%%%%%%%%%%%%%%%%%%%%%%%%%%%%%%%%%%%%%%%%%%%%%%%%%%%%%%%%%%%%%%%%%%%%%%

\section{Recruiting Members}

The primary responsibility for recruiting new team members lies with the Area Managers.  In those circumstances where a new Operating Area is being launched or there is no Area Manager, the Amateur Radio Coordinator assumes a direct responsibility for team member recruitment.

%%%%%%%%%%%%%%%%%%%%%%%%%%%%%%%%%%%%%%%%%%%%%%%%%%%%%%%%%%%%%%%%%%%%%%%%

\section{Spotter Outreach}

Area Managers are encouraged to attend Spotter training classes held in their assigned Operating Area(s).  In the event an Area Manager cannot attend, he is expected to notify the Amateur Radio Coordinator as soon as possible to that appropriate coverage can be obtained.

While attending Spotter training classes, the person representing the team should:

\begin{enumerate} 
\item Introduce himself to the NWS employees conducting the session prior to its start.
\item Work with the NWS employees to find a convenient time to address the audience.
\item Talk briefly about what amateur radio is and what role it plays in the SKYWARN program.
\item Provide local frequency information.
\item Share contact information for those who are licensed and want to get involved in the ham radio team or who are unlicensed and are considering getting licensed.
\item Answer any audience questions.
\item Remain available after the class to talk one-on-one with class participants.
\end{enumerate}

The person representing the team is expected to dress appropriately and provide a professional appearance and attitude on behalf of the team at all times.

%%%%%%%%%%%%%%%%%%%%%%%%%%%%%%%%%%%%%%%%%%%%%%%%%%%%%%%%%%%%%%%%%%%%%%%%

\section{Club Outreach}

In order to remain visible and maintain a good relationship with the amateur radio community it is imperative that both the Amateur Radio Coordinator and Area Managers frequently engage in basic outreach work throughout their assigned Operating Area(s).  This outreach work can be anything from simply being present at club meetings to put a face on SKYWARN, up to full outreach presentations that discuss how club members can support the SKYWARN program with their reports and by volunteering as Net Control Operators.

Whenever feasible, Area Managers and the Amateur Radio Coordinator are encouraged to maintain a personal membership with as many local amateur radio clubs as possible and participate in as many club activities as possible as a means of developing a mutual support system between the club and SKYWARN.

%%%%%%%%%%%%%%%%%%%%%%%%%%%%%%%%%%%%%%%%%%%%%%%%%%%%%%%%%%%%%%%%%%%%%%%%

\section{Public Outreach}

Public outreach opportunities most frequently come in the form of outreach presentations to Community Emergency Response Teams (CERT) and a presence at local hamfests.  Many hamfests offer no-charge tables to non-selling organizations such as SKYWARN, and the team maintains a display board and a collection of printable materials which can be used for any sort of public outreach opportunity.

Area Managers are encouraged to get SKYWARN involved in these sorts of activities and should work through the Amateur Radio Coordinator to get NWS employees involved whenever possible.

%%%%%%%%%%%%%%%%%%%%%%%%%%%%%%%%%%%%%%%%%%%%%%%%%%%%%%%%%%%%%%%%%%%%%%%%

\section{NWS Outreach}

NWS Outreach opportunities occur whenever there is an interaction between a SKYWARN amateur radio team member and an NWS employee.  In particular, there should be an effort to familiarize new forecasters with the team and our procedures for activation.  Journeyman Forecasters cycling into the WFO represent a prime opportunity for education on amateur radio and how it supports NWS operations through the SKYWARN program.

The team maintains an NWS Staff Outreach presentation which can be shared with new WFO employees through the SKYWARN Program Manager.  Additional outreach opportunities can be coordinated between the Amateur Radio Coordinator and the SKYWARN Program Manager.

%%%%%%%%%%%%%%%%%%%%%%%%%%%%%%%%%%%%%%%%%%%%%%%%%%%%%%%%%%%%%%%%%%%%%%%%

\section{Partnerships}

The Amateur Radio Coordinator is encouraged to maintain a close relationship with our emergency service partners in several key organizations:

\begin{enumerate}
\item Amateur Radio Emergency Services (ARES).
\item Virginia Department of Emergency Management (VDEM) Amateur Radio Communications Auxiliary (ARCA).
\item Local emergency management personnel.
\end{enumerate}

These partnerships can be accomplished by:

\begin{enumerate} 
\item Serving as an ARES Official Emergency Station (OES) as SKYWARN Liaison.
\item Participating in VDEM-ARCA meetings, training, and drills.
\item Sharing report data with decision makers and emergency managers.
\item Involving local emergency management personnel in the SKYWARN training process.
\end{enumerate}

It is absolutely critical that these relationships be developed and nurtured to ensure cooperation and interoperability during emergency situations.

%%%%%%%%%%%%%%%%%%%%%%%%%%%%%%%%%%%%%%%%%%%%%%%%%%%%%%%%%%%%%%%%%%%%%%%%
%%%%%%%%%%%%%%%%%%%%%%%%%%%%%%%%%%%%%%%%%%%%%%%%%%%%%%%%%%%%%%%%%%%%%%%%
%%%%%%%%%%%%%%%%%%%%%%%%%%%%%%%%%%%%%%%%%%%%%%%%%%%%%%%%%%%%%%%%%%%%%%%%
%%%%%%%%%%%%%%%%%%%%%%%%%%%%%%%%%%%%%%%%%%%%%%%%%%%%%%%%%%%%%%%%%%%%%%%%
%%%%%%%%%%%%%%%%%%%%%%%%%%%%%%%%%%%%%%%%%%%%%%%%%%%%%%%%%%%%%%%%%%%%%%%%
%%%%%%%%%%%%%%%%%%%%%%%%%%%%%%%%%%%%%%%%%%%%%%%%%%%%%%%%%%%%%%%%%%%%%%%%

\chapter{Training}

To maintain an effective team of communicators and to meet service commitments, a solid training program is critical.  This section outlines the fundamental concepts to be included in the training process for various SKYWARN amateur radio team positions.

Some training and experience requirements may be relaxed in new Operating Areas.

%%%%%%%%%%%%%%%%%%%%%%%%%%%%%%%%%%%%%%%%%%%%%%%%%%%%%%%%%%%%%%%%%%%%%%%%

\section{Spotter Training Requirement}

All SKYWARN amateur radio team members must have valid SKYWARN Spotter certification and must be able to furnish their Spotter ID.  Since Spotter ID's are no longer issued by the National Weather Service at the time of training, this information must be obtained via e-mail correspondence with the SKYWARN training coordinator at the Wakefield WFO.  Spotter training is valid for three years, by which time the Spotter must attend either Basic or Advanced SKYWARN Spotter training to remain certified.

The Wakefield WFO considers training ``portable'' between the Wakefield, Newport/Morehead City, Raleigh, Blacksburg, Sterling, and Mt. Holly offices.  Spotter ID's or certificates of completion obtained within the last three years from any of these offices will be considered valid for purposes of fulfilling this requirement.

The Spotter training requirement is customarily waived for team members whose service to SKYWARN is purely in a supporting role, for example, IT staff and those in technical support and equipment maintenance positions.

%%%%%%%%%%%%%%%%%%%%%%%%%%%%%%%%%%%%%%%%%%%%%%%%%%%%%%%%%%%%%%%%%%%%%%%%

\section{Licensing Requirement}

Due to the nature of the team's operations, all members must hold a valid amateur radio license of the class appropriate to their position as specified in this manual and must remain in good standing with the Federal Communications Commission for the duration of their participation in this program.

%%%%%%%%%%%%%%%%%%%%%%%%%%%%%%%%%%%%%%%%%%%%%%%%%%%%%%%%%%%%%%%%%%%%%%%%

\section{NIMS and ICS}

To ensure interoperability with other emergency service organizations, SKYWARN volunteers wishing to serve in the Responder or Area Manager roles must complete the FEMA IS.100, IS.200, IS.700, and IS.800 training on the National Incident Management System (NIMS) and Incident Command System (ICS).  This training is available online and certificates are provided in PDF format which should be kept on file in the official team records.

The NIMS and ICS training requirement is required of all Responders as of December 31, 2011.

%%%%%%%%%%%%%%%%%%%%%%%%%%%%%%%%%%%%%%%%%%%%%%%%%%%%%%%%%%%%%%%%%%%%%%%%

\section{Net Control Operator Training}\label{nco-training}

All amateur radio team members serving in the Net Control Operator (NCO), Responder, Net Manager, Area Manager, or Amateur Radio Coordinator roles must complete a Net Control Operator training course covering a minimum of the following topics:

\begin{enumerate}
\item Overview of SKYWARN and the amateur radio team.
\item Understanding of when and how SKYWARN activates.
\item Calling a net and keeping track of check-ins.
\item Identifying severe weather.
\item Asking probing questions to fully develop a spotter report.
\item Identifying and properly handling potentially fraudulent reports.
\item Dealing with interference and malicious activity.
\item \nameref{handling-sensitive}
\item Dealing with periods of high traffic.
\item Coexisting with other nets on the same frequency.
\item Properly logging all reports received into the SKYWARN Report Management System (RMS).
\item Determining which reports need to be manually relayed to the NWS WFO and properly relaying those reports.
\item When and how to switch to a directed net.
\item Dealing with repeater outages and other technical issues.
\item Working with linked nets via IRLP and Echolink.
\item Securing the net.
\end{enumerate}

Net Control Operator training is valid for a period of three years but more frequent specialized training may be required in the event of significant changes in operating procedure.

Reserve Net Control Operators --- those NCO's who have not logged at least one report in an 18-month period --- will be required to complete the Net Control Operator training program annually.

Training may be delivered either in a classroom setting, online, or via teleconference, and the Amateur Radio Coordinator should determine if a post-training assessment is required to validate trainees have learned the necessary skills prior to certifying or re-certifying them as Net Control Operators.

%%%%%%%%%%%%%%%%%%%%%%%%%%%%%%%%%%%%%%%%%%%%%%%%%%%%%%%%%%%%%%%%%%%%%%%%

\section{Responder Training}

Responders are required to complete all \nameref{nco-training} as well as a Responder training program covering the following topics:

\begin{enumerate}
\item Understanding when and how to respond to the WFO.
\item Preparing for a response to the WFO, including what to pack.
\item Gaining access to the WFO and the SKYWARN Radio Desk.
\item Signing in and out of the SKYWARN Radio Desk.
\item Storage of personal belongings at the WFO.
\item Operation of all radio equipment at the radio desk.
\item Locating and installing reserve radio equipment.
\item Locating SKYWARN reference materials on the SKYWARN computer.
\item Understanding and controlling linked repeater systems.
\item Identification and use of Auxiliary SKYWARN Repeaters.
\item Working with NWS employees to stay abreast of severe weather threats and specific areas NWS employees need reports from.
\item Properly and effectively relaying reports from nets to NWS employees.
\item Utilizing APRS and Winlink to manually retrieve log entries from the SKYWARN RMS.
\end{enumerate}

Responder training should be conducted one-on-one with new Responders on-site at the WFO with the coordination of the SKYWARN Program Manager.  Responder training is valid for three years unless considerable changes in equipment or operating procedure dictate more frequent training.

%%%%%%%%%%%%%%%%%%%%%%%%%%%%%%%%%%%%%%%%%%%%%%%%%%%%%%%%%%%%%%%%%%%%%%%%

\section{Area Manager Training}

New Area Managers should receive training on the following topics:

\begin{enumerate}
\item Understanding the various SKYWARN activation assessment and planning tools.
\item Understanding activation time frames and net staffing requirements.
\item Utilizing the Team Roster to locate Net Controllers to staff a net.
\item Working with other Area Managers to find additional Net Controllers.
\item How to give an effective outreach presentation and utilize all available outreach resources.
\item Recruiting new Net Controllers.
\item Working with the SKYWARN Focal Point(s) to arrange spotter training classes.
\item Effectively communicating with Net Controllers and other SKYWARN team members.
\item Effective leadership skills.
\end{enumerate}

Training should be one-on-one and the depth of the training could vary widely depending on the Area Manager's background and leadership experience.

%%%%%%%%%%%%%%%%%%%%%%%%%%%%%%%%%%%%%%%%%%%%%%%%%%%%%%%%%%%%%%%%%%%%%%%%

\section{Training Records}

Area Managers are responsible for gathering up-to-date training information on a quarterly basis and forwarding this to the Amateur Radio Coordinator.  Area Managers are responsible for ensuring everyone on their team maintains current training as required by their position and the current training curriculum. As team members obtain additional or refresher training, the Area Manager shall report this to the Amateur Radio Coordinator for documentation in the team's records.

%%%%%%%%%%%%%%%%%%%%%%%%%%%%%%%%%%%%%%%%%%%%%%%%%%%%%%%%%%%%%%%%%%%%%%%%

\section{Ongoing Training and Re-certification}

Periodic refresher training shall be required at least once every three years for all training elements with the exception of the NIMS/ICS components, for which refresher training shall be required based on FEMA guidelines.

%%%%%%%%%%%%%%%%%%%%%%%%%%%%%%%%%%%%%%%%%%%%%%%%%%%%%%%%%%%%%%%%%%%%%%%%
%%%%%%%%%%%%%%%%%%%%%%%%%%%%%%%%%%%%%%%%%%%%%%%%%%%%%%%%%%%%%%%%%%%%%%%%
%%%%%%%%%%%%%%%%%%%%%%%%%%%%%%%%%%%%%%%%%%%%%%%%%%%%%%%%%%%%%%%%%%%%%%%%
%%%%%%%%%%%%%%%%%%%%%%%%%%%%%%%%%%%%%%%%%%%%%%%%%%%%%%%%%%%%%%%%%%%%%%%%
%%%%%%%%%%%%%%%%%%%%%%%%%%%%%%%%%%%%%%%%%%%%%%%%%%%%%%%%%%%%%%%%%%%%%%%%
%%%%%%%%%%%%%%%%%%%%%%%%%%%%%%%%%%%%%%%%%%%%%%%%%%%%%%%%%%%%%%%%%%%%%%%%

\chapter{Activating SKYWARN}

%%%%%%%%%%%%%%%%%%%%%%%%%%%%%%%%%%%%%%%%%%%%%%%%%%%%%%%%%%%%%%%%%%%%%%%%

\section{Service Level Commitments}

The team is ``on standby'' 24 hours a day, 7 days a week and should be able to provide emergency communications services and collection of spotter reports any time of the day or night.

To better utilize its human resources the team has implemented three Service Level Commitments which specify the circumstances under which routine report collection services will be provided to the NWS.

In general, amateur radio support services will be provided between 6 AM and 10 PM local time.  A Nighttime SLC applies to operation outside these hours, and there is a separate SLC for the SKYWARN Radio Desk.

\subsection{Daytime Service Level Commitment}\label{daytime-slc}

The team will make itself available to the NWS upon request under any weather conditions. It will self-activate between the hours of 6 am and 10 pm local time under any one or more of the following conditions:

\begin{enumerate}
\item Severe Thunderstorm Watch or Tornado Watch meeting one or more of the following criteria:
 \begin{enumerate}
 \item SPC Enhanced Resolution Thunder Risk at 40\% or greater
 \item SPC Day 1 Convective Outlook Damaging Wind risk at or above 15\%
 \item SPC Day 1 Convective Outlook Severe Hail risk at or above 15\%
 \item SPC Day 1 Convective Outlook Tornado risk at or above 5\%.
 \end{enumerate}
\item Any PDS Tornado Watch.
\item Any Tornado Warning.
\item Certain severe warned thunderstorms where the storms either demonstrate a history of reported damage or show a high likelihood of producing reportable damage, with or without a watch product in place (generally local discretion).
\item Any winter weather event capable of producing icing conditions or sufficient snow or sleet accumulations so as to make travel hazardous.
\item Tropical weather events including tropical storms and hurricanes.
\item Upon request from the NWS.
\item At the discretion of local SKYWARN Net Control Operators and Area Managers.
\end{enumerate}

When any of these conditions are met, SKYWARN nets will prepare to activate and will go on the air once severe or potentially severe weather is impacting or about to impact the area, provided such weather conditions are occurring between the hours of 6 AM and 10 PM local time.  The team will activate anytime there is a formal request from the NWS.

\subsection{Nighttime Service Level Commitment}\label{nighttime-slc}

Overnight activations of SKYWARN nets, between 10 PM and 6 AM local time, will occur only under the following conditions:

\begin{enumerate}
\item Severe Thunderstorm Watch related to storms with a history of reportable damage, with an SPC Enhanced Resolution Thunder Risk of 40\% or greater, and meeting one or more of the following criteria:
 \begin{enumerate}
 \item SPC Day 1 Convective Outlook Damaging Wind risk at or above 30\%.
 \item SPC Day 1 Convective Outlook Severe Hail risk at or above 30\%.
 \item SPC Day 1 Convective Outlook Tornado risk at or above 10\%.
 \end{enumerate}
\item Any Tornado Watch meeting one or more of the following criteria:
 \begin{enumerate}
 \item SPC Day 1 Convective Outlook Damaging Wind risk at or above 30\%.
 \item SPC Day 1 Convective Outlook Severe Hail risk at or above 30\%.
 \item SPC Day 1 Convective Outlook Tornado Risk at or above 10\%.
 \end{enumerate}
\item Any PDS (Particularly Dangerous Situation) Tornado Watch.
\item High-impact tropical weather events such as a hurricane making landfall or passing very near the coast.
\item Upon request from the NWS.
\item At the discretion of local SKYWARN Net Control Operators and Area Managers.
\end{enumerate}

It bears repeating that any formal request from the NWS for overnight support fully overrides any specific weather criteria stated herein.

\subsection{SKYWARN Radio Desk Service Level Commitment}\label{radio-desk-slc}

In many high-impact weather events such as tornado outbreaks and tropical weather systems it is important that we have a qualified Responder at the SKYWARN Radio Desk.  History has shown that the amount of radio traffic drops off sharply overnight except in the worst of weather conditions, and attempting to keep SKYWARN on the air overnight is usually counterproductive.

Any combination of weather conditions and travel distance may be a deciding factor in whether to remove a Responder from the WFO overnight.  In general at least one Responder should remain on-site overnight when significant weather conditions are forecast to continue into the morning in case NWS employees require sudden overnight amateur radio assistance.

To better balance the need for sleep with the need for ``on-call'' Responders in the WFO, the SKYWARN Radio Desk Service Level Commitment has been developed to specify the hours during which Responder support will generally be provided at the SKYWARN Radio Desk.

A period of ``quiet hours'' has been established, running from 11 PM to 5 AM local time.  During this period the Responder will remain at the WFO but will be unavailable for NWS support.  A later waking time may be arranged with NWS employees prior to going off duty, but is subject to change at any time.

The exact quiet hours will vary from one activation to the next, but these are the general guidelines specifying the ideal time frame in which to suspend Responder support at the radio desk.

At any time the Responder may be summoned back to the Radio Desk by any NWS employee should an urgent situation arise.

\subsection{Handling Nets In Progress at SLC Close}

Any SKYWARN Net which is on the air at the close of the SLC window is not automatically suspended.  Provided there are stations checked in, the net should be continued until such time that it is reasonable to shut it down, or until the Net Control Operator's personal availability or stamina requires a net be closed.  Ultimately it us up to the Area Manager to determine the most appropriate time to shut down a late evening net unless a request for longer operation has been received from the NWS.

\subsection{SLC Impacts on IRLP and Echolink SuperNets}

Whenever possible, the Echolink Conference and IRLP Reflector would be utilized except during the WX4AKQ quiet hours.  If a net is active on an Echolink or IRLP-equipped repeater, and if no other Net Controller is monitoring the Conference or Reflector, that net should be linked in to take reports from other stations which may connect.

In other words, every effort should be made to ensure that the Echolink Conference and IRLP Reflector are monitored during the \nameref{daytime-slc} hours specified in this manual, unless the SKYWARN Radio Desk is also active, in which case the \nameref{radio-desk-slc} specified separately in this manual shall dictate the required monitoring hours.

%%%%%%%%%%%%%%%%%%%%%%%%%%%%%%%%%%%%%%%%%%%%%%%%%%%%%%%%%%%%%%%%%%%%%%%%

\section{Activation Forecasting}

\subsection{Amateur Radio Coordinator Role in Forecasting}

The Amateur Radio Coordinator is responsible for monitoring all available products from the NWS to determine the general need for SKYWARN activation and amateur radio support over the next 24 to 36 hours.  Some recommended products are:

\begin{enumerate}
\item SPC Day 1 and Day 2 Convective Outlooks (SWO).
\item HPC Day 1 and Day 2 Quantitative Precipitation Forecast (QPF).
\item HPC Day 1 and Day 2 QPF Excessive Rainfall Discussion (QPFERD).
\item HPC winter precipitation products.
\item NHC tropical products.
\item WFO AKQ Area Forecast Discussion (AFD).
\item WFO AKQ Hazardous Weather Outlook (HWO).
\item Direct guidance from NWS employees.
\end{enumerate}

In the short term (4-12 hours) the HWO and discussion with NWS forecasters are some of the best tools for determining the local need for SKYWARN support, in addition to these products:

\begin{enumerate}
\item SPC Mesoscale Discussions (MCD).
\item Severe Thunderstorm Watches.
\item Tornado Watches.
\item WFO AKQ Hazardous Weather Outlook (HWO).
\item Local Storm Reports (LSR) from WFO's and SPC.
\item Direct guidance from NWS employees.
\end{enumerate}

Based on this guidance and the activation criteria specified in this manual, the Amateur Radio Coordinator should determine the potential needs for SKYWARN net activation and communicate this with the leadership team.

\subsection{Coordination with NWS Employees}

A decision to activate local nets will generally be an easy one when using the criteria and resources outlined above.  Usually there will be no need to contact NWS employees for guidance on whether to activate local nets, and there is no need to notify the WFO of local net activations.

NWS employees should be consulted on the activation of the SKYWARN Radio Desk.  For the majority of severe weather events there will be no substantial benefit to activating the SKYWARN Radio Desk.  However, for large-scale or outbreak type weather events that have the potential to produce widespread damage or that will impact a large portion of the CWA simultaneously, activation of the SKYWARN Radio Desk is a possibility.

A phone call to the WFO several hours before the expected onset of severe weather will allow for discussion of the threat and will help make a decision to activate the SKYWARN Radio Desk. If either the Amateur Radio Coordinator or NWS employees determine the activation of the Radio Desk would be beneficial, the Amateur Radio Coordinator will use the Responder roster to develop a list of volunteers to staff the Radio Desk and should provide a list of names to the NWS employees as soon as possible.

Responders should never be sent to the SKYWARN Radio Desk without a request from NWS.

\subsection{SKYWARN Risk Assessments}\label{risk-assessments}

The SKYWARN Risk Assessment (RA) is an internal bulletin produced by the Amateur Radio Coordinator which provides:

\begin{enumerate}
\item A headline with a brief description of the expected severe weather threat.
\item A one- or two-paragraph introduction to the severe weather threat.
\item The expected onset of severe weather.
\item The Operating Areas expected to be impacted by the event.
\item The expected duration of the event.
\item A Spotter Activation Outlook.
\item A Communications Team Action Message outlining the need for activation of local nets and the SKYWARN Radio Desk.
\end{enumerate}

The RA should be distributed as early as possible and at least once every four hours while a severe weather threat exists.

Area Managers should use the information in the RA to assemble a team of Net Control Operators to staff the local SKYWARN net, while the Amateur Radio Coordinator prepares to staff the SKYWARN Radio Desk if necessary.

%%%%%%%%%%%%%%%%%%%%%%%%%%%%%%%%%%%%%%%%%%%%%%%%%%%%%%%%%%%%%%%%%%%%%%%%

\section{Activation Procedures}

\subsection{Area Manager Role in Activations}

Area Managers are responsible for using the Risk Assessment and various NWS products as guidance in developing a schedule of Net Control Operators to staff the local SKYWARN net.  The Area Manager should maintain information on individual Net Control availability and capabilities.

In the event that an insufficient number of Net Control Operators can be found to staff the net, neighboring Area Managers should be consulted regarding the sharing of Net Control resources across Operating Area boundaries.  If linking technologies such as Echolink or IRLP are available, they may be used to bridge two Operating Areas together to overcome a staffing deficiency.

In no uncertain terms, the Area Manager is personally responsible for ensuring the local net is on the air when it needs to be.  If no Net Control resources are available within the Operating Area or an adjacent Area and no relief can be obtained through linking technologies, the Area Manager is required to run the net personally.

Area Managers should report to the Amateur Radio Coordinator when nets are on the air and when they are secured, and should also promptly communicate any problems or concerns that arise during the net.

\subsection{Team Activation Notifications}\label{team-notification}

The Amateur Radio Coordinator should send out an Activation Notification via e-mail or other electronic system to notify the team when SKYWARN is pending activation, activated, or deactivated in any Operating Area.  At a minimum this should be sent to the impacted Operating Area and all Area Managers.

\subsection{Notification of Partners}

Notification of SKYWARN activation should be made to the appropriate SKYWARN partners including state and local EOC's in impacted areas, local emergency managers, and others who have requested to be notified of SKYWARN amateur radio activations.  Notification may be made electronically or over the air. The activation notification should include the anticipated date and time of activation (if future) and the frequencies on which SKYWARN will be active.

\subsection{NWS Notification}

It is not necessary to notify NWS employees when local SKYWARN nets are being activated or deactivated, unless they have requested such notification ahead of time.

Since coordination with the NWS is required for activation of the SKYWARN Radio Desk, NWS employees will usually be aware of an amateur radio response to the WFO.  However, the Amateur Radio Coordinator should make a courtesy call to the WFO prior to the arrival of the Responders and should provide the estimated time of arrival and headcount.  A list of Responder names should also be available in case it is requested by the NWS employees on duty.

\subsection{Deactivation}

Local SKYWARN nets may be deactivated once the severe weather threat has passed or upon closure of the \nameref{daytime-slc} window, described on page \pageref{daytime-slc}.  Upon deactivation, all persons and organizations who received an Activation Notification should receive notice of deactivation.  This notice should include any anticipated reactivation time.  For more information on these notifications, see \nameref{team-notification} on page \pageref{team-notification}.

Deactivation of the SKYWARN Radio Desk will be at the discretion of NWS employees.

%%%%%%%%%%%%%%%%%%%%%%%%%%%%%%%%%%%%%%%%%%%%%%%%%%%%%%%%%%%%%%%%%%%%%%%%
%%%%%%%%%%%%%%%%%%%%%%%%%%%%%%%%%%%%%%%%%%%%%%%%%%%%%%%%%%%%%%%%%%%%%%%%
%%%%%%%%%%%%%%%%%%%%%%%%%%%%%%%%%%%%%%%%%%%%%%%%%%%%%%%%%%%%%%%%%%%%%%%%
%%%%%%%%%%%%%%%%%%%%%%%%%%%%%%%%%%%%%%%%%%%%%%%%%%%%%%%%%%%%%%%%%%%%%%%%
%%%%%%%%%%%%%%%%%%%%%%%%%%%%%%%%%%%%%%%%%%%%%%%%%%%%%%%%%%%%%%%%%%%%%%%%
%%%%%%%%%%%%%%%%%%%%%%%%%%%%%%%%%%%%%%%%%%%%%%%%%%%%%%%%%%%%%%%%%%%%%%%%

\chapter{General Net Procedures}

%%%%%%%%%%%%%%%%%%%%%%%%%%%%%%%%%%%%%%%%%%%%%%%%%%%%%%%%%%%%%%%%%%%%%%%%

\section{Before the Net}

Net Control should perform a basic equipment check prior to getting on the air, ensuring that their radio equipment is functioning properly and that they have access to the SKYWARN Report Management System (RMS).  Paper log sheets and scripts should be printed, along with a paper copy of the Team Roster and a current SKYWARN frequency list.

Net Control needs to ensure that the repeater is also in good working order and is not in use by another agency. 

%%%%%%%%%%%%%%%%%%%%%%%%%%%%%%%%%%%%%%%%%%%%%%%%%%%%%%%%%%%%%%%%%%%%%%%%

\section{Acquiring the Frequency}

In accordance with good amateur radio practice, Net Control should identify and make a couple of inquiries as to whether the repeater is in use.  If a conversation is in progress, Net Control should politely break in and mention the need to start a SKYWARN net on the frequency.  The stations should be encouraged to surrender the frequency as soon as possible, though there is nothing that can force the stations to stop talking.  Most stations will gladly give up the frequency within a minute or two.  If the frequency cannot be used due to an ongoing conversation, the Area Manager should be consulted for guidance.

%%%%%%%%%%%%%%%%%%%%%%%%%%%%%%%%%%%%%%%%%%%%%%%%%%%%%%%%%%%%%%%%%%%%%%%%

\section{NCO Station Identification}\label{nco-station-id}

Nets should not use the WX4AKQ call sign unless they are being conducted directly from the SKYWARN Radio Desk.  Instead, each individual Net Control Operator should use his/her own call sign. More information on use of the WX4AKQ call sign can be found in \nameref{wx4akq-usage} on page \pageref{wx4akq-usage}.

%%%%%%%%%%%%%%%%%%%%%%%%%%%%%%%%%%%%%%%%%%%%%%%%%%%%%%%%%%%%%%%%%%%%%%%%

\section{Net Scripts}\label{scripts}

\subsection{Official Scripts}

The team maintains one or more scripts for use under various conditions.  One universal script may be used year-round, or there may be separate scripts for convective and winter weather activations, with additional scripts for informal nets and directed nets.

Net Control Operators are encouraged to use these scripts,  especially until they are comfortable running the net.  Printed copies of the current scripts should be kept – along with log sheets, a printed copy of the Roster, and current SKYWARN frequencies – at all locations from which the Net Control Operator might be operating SKYWARN nets (home, car, work, etc.)

Official scripts will be posted on the SKYWARN Operations Portal web site and copies should be maintained at the SKYWARN Radio Desk.

\subsection{Unofficial Scripts}

There is no requirement that the scripts be used. The Operating Areas and even individual Net Controllers may adjust or even completely re-write their scripts to suit their own preferences.

\subsection{Mandatory Script Content}\label{mandatory-content}

All scripts – both official and unofficial – must identify the net as the Wakefield SKYWARN Spotter Net, must identify the net controller and the purpose of the net, must provide a reminder of \nameref{reporting-criteria} applicable to the current weather scenario, and must provide a periodic recap of current watches, warnings, and statements.

%%%%%%%%%%%%%%%%%%%%%%%%%%%%%%%%%%%%%%%%%%%%%%%%%%%%%%%%%%%%%%%%%%%%%%%%

\section{Informal vs. Directed Nets}

\subsection{Informal Nets Defined}
An informal net is used for the majority of SKYWARN activations.  Informal nets allow for most efficient use of the repeater by allowing a free flow of information and by keeping the repeater open for regular amateur radio communications.

During informal nets, Net Control does maintain a list of check-ins and may periodically call on specific stations based on location.  However, stations do not need to request permission to transmit from Net Control and may communicate amongst themselves with or without being checked in to the net.

Periodically (generally every 10 to 15 minutes) Net Control will read the script, call for check-ins and check-outs, and provide a recap of the severe weather threat along with any watches, warnings, and statements as specified in the ``\nameref{mandatory-content}'' on page \pageref{mandatory-content}.

\subsection{Directed Nets Defined}

Directed nets are used only under either of two conditions:

\begin{enumerate}
\item When a Tornado Warning is active for the net area.
\item When the amount of net traffic is so great as to require a more formal level of traffic management. This is common during winter weather events when many dozens of stations pile up to call in their snowfall measurements to Net Control.
\end{enumerate}

During directed nets, all transmissions are at the direction of Net Control. Stations check in with Net Control and wait to be recognized.  Stations should be asked during the check-in process if they have any specific report for SKYWARN.  Stations with reports to pass should be called at the conclusion of check-ins unless there is emergency traffic which must be handled sooner.

As traffic is handled, stations should be asked if they wish to remain checked in to the net.

Stations may still use the repeater to pass traffic to one another, provided it is done at the direction of Net Control.  It is imperative that Net Control understand that not all stations will be familiar with the concept of a directed net or the procedures that go along with one, so some stations will take advantage of pauses in net activity to call other stations.  If the discussion goes on much longer than simply coordinating a phone call or a switch to a different frequency, Net Control should politely and tactfully remind the stations that a directed SKYWARN net is in progress and they should use an alternate frequency for their discussion.

\subsection{Transitioning Net Formats}

If a net is operating as an informal net and either a tornado warning is issued or the amount of traffic passing through net control warrants a shift to a directed net, Net Control should use this procedure to make the change:

\begin{enumerate}
\item Announce that due to the tornado warning or heavy traffic volume, SKYWARN is transitioning into a directed net. If the switch is due to a tornado warning, read a quick recap of the warning.
\item Explain that as part of the directed net all transmissions will be at the direction of net control and all stations on frequency are respectfully asked to please hold all non-SKYWARN and non-emergency traffic until such time that the tornado warning expires or traffic subsides to the point the informal net can resume.
\item Briefly list the call signs of stations already checked in and ask if there are any other stations wishing to check in.
\item Ask stations that have traffic to call for SKYWARN Net Control and then promptly handle that traffic.
\item Provide another recap of the tornado warning, if any.
As soon as the tornado warning expires or when traffic volume subsides, the net should be reverted back to an informal net and an announcement should be made that transmissions no longer require net control authorization and the frequency is once again open for routine traffic.
\end{enumerate}

%%%%%%%%%%%%%%%%%%%%%%%%%%%%%%%%%%%%%%%%%%%%%%%%%%%%%%%%%%%%%%%%%%%%%%%%

\section{Net Check-Ins}

\subsection{Purpose of Check-Ins}

Asking stations to check in to the net provides a means of keeping track of who is on the air and where they are located.  This gives Net Control a way to go to specific areas for reports as appropriate for the weather situation or at the request of NWS employees, and it also allows Net Control to keep tabs on the welfare of stations in the path of severe weather.

\subsection{Information Collected}

Net Control should make a note of the call sign, first name, and location of stations as they check in.  If a station is mobile, direction of travel should also be noted.

\subsection{Welfare Check-ups}

Net Control should keep an eye on radar and other NWS products and should keep stations appraised of the severe weather threats.  Stations in the path of severe weather should be notified and encouraged to take shelter and then check back in with Net Control as soon as the threat passes.

Area Managers should be notified if a station in imminent danger goes off the air and does not check back in, and should investigate further.

\subsection{Roll Calls}

During longer activations, about once every 60 to 90 minutes, Net Control should run down the list of check-ins and ask each station to come back with an acknowledgment that they are still on frequency and wish to either remain on the net or be checked out (stations may check out at any time and do not need to wait for the roll call.)  Stations that do not answer should be called again at the end of the first roll call, and then removed from the list if there is still no response.

\subsection{Handing Off Net Control}

When handing off the net from one Net Controller to another, the list of current check-ins and the time of the last roll call should be provided to the incoming Net Controller.  This can be done over the air, online, or through any other method available.  This will prevent hassling stations to check in again just because of a change in Net Controllers.

\subsection{Record Keeping Requirements}

There is no need to keep a list of check-ins past the end of a net. Neither the details nor a count of check-ins needs to be documented in the team records.  No log entries should be made for simple check-ins or check-outs.

%%%%%%%%%%%%%%%%%%%%%%%%%%%%%%%%%%%%%%%%%%%%%%%%%%%%%%%%%%%%%%%%%%%%%%%%

\section{Taking Reports}

\subsection{Reportable Events}

``Reportable events'' are any weather events which meet the NWS established \nameref{reporting-criteria} provided on page \pageref{reporting-criteria}.

\subsection{All Reports Taken}

One of the goals of the net scripts (described starting on page \pageref{scripts}) is to provide a reminder of what constitutes a reportable severe weather event.  That is, what kinds of reports the SKYWARN net is looking for.  However, all reports taken into a SKYWARN net must be accepted and logged.

Any SKYWARN team member who knowingly and willfully engages in the collection of Spotter reports via amateur radio and fails to log those reports in the electronic logging system in a timely manner is subject to dismissal from the team.  There are \nameref{two-exceptions}.

\subsection{Information Collected}

Reports should include the call sign and name of the Spotter, the location of the event, the date and time of the event (if known, otherwise estimate if possible), the date and time of the report, and the details of the report.  If the station calling in the report is a trained Spotter, it should be indicated on the report.  Since NWS does not issue Spotter ID numbers any longer, it is not necessary to ask for a Spotter ID or record it in the report.

\subsection{Logging Requirements}

All reports received into the SKYWARN net must be logged, regardless of whether the report meets SKYWARN reporting criteria, with two exceptions, noted below.  The SKYWARN Report Management System (RMS) must be used for all logging unless Internet access is not available to Net Control.

If no Internet access is available to Net Control, or if RMS is down, paper log sheets should be used. Once service is restored, logs must be manually entered into RMS.  When entering this information care should be exercised to avoid re-notifying NWS with a duplicate report out of RMS, and attention to detail is necessary to ensure that the dates and times are entered correctly.

Net Controllers who require assistance getting their paper logs entered into RMS can mail or fax them to their Area Manager who will put the data into the permanent logs.

\subsection{Two Exceptions to Logging Requirements}\label{two-exceptions}

There are two types of report which should not be logged:

\textbf{Reports received via certain third-party sources.}  This includes reports obtained from radio or television media outlets, overheard on a scanner or public safety radio, or rumor.  Reports from media and public safety sources are already collected by NWS through other channels.

\textbf{Reports not called in to the net.}  This includes reports received via telephone from friends, family, neighbors, or other parties, and reports originating from the Net Control Operator himself.  The log system is intended for collection of reports collected from another station via amateur radio only, and other reporting mechanisms, such as e-mail, telephone, e-Spotter, or social media should be used for all other report types.

%%%%%%%%%%%%%%%%%%%%%%%%%%%%%%%%%%%%%%%%%%%%%%%%%%%%%%%%%%%%%%%%%%%%%%%%

\section{Handling Sensitive Information}\label{handling-sensitive}

Occasionally a Net Controller will be faced with the task of taking a report of an injury or death due to weather.  It is extremely important that this information be handled carefully.  Here are the guidelines for handling sensitive information:

\begin{enumerate}
\item Only request the minimum amount of information required to form the report.  The location (address, nearby intersection or landmark), number and type of injuries (for example, two adults with injuries to the legs and one child trapped under debris) will suffice.
\item If this is an initial report for an injury or fatality, gather as much information as possible and immediately relay the information to emergency officials by telephone or radio. Call 911 and request a transfer to the appropriate county or city.  When filing the report in RMS, include all details collected.
\item If RMS is down or unavailable and the report must be relayed over the air to the NWS, only give the minimum information necessary to convey the report.  Do not give names or other specifics unless requested by NWS, and upon receiving such a request it should go by telephone or APRS to avoid sending the information ``in the clear'' for the public to hear.
\item Do not dwell on the situation.  Keep the net moving.  If Net Control is personally (emotionally) affected by the report, they should contact their Area Manager or another Net Controller for temporary relief.
\item Do not discuss the information with anyone not immediately a party to the transmission.  Of course the exception is law enforcement and emergency crews responding to the incident.
\end{enumerate}

In other words, get every piece of information necessary to summon aid if still needed, but only relay the bare minimum information over the air to the NWS.

%%%%%%%%%%%%%%%%%%%%%%%%%%%%%%%%%%%%%%%%%%%%%%%%%%%%%%%%%%%%%%%%%%%%%%%%

\section{Relaying Reports}\label{relaying}

\subsection{Sending Reports to NWS}

RMS will automatically send reports to NWS via e-mail as soon as the reports are released from the system.  If the WFO's Internet connectivity is down, these reports will need to be relayed by telephone or some other means.  Only reports that meet \nameref{reporting-criteria}, included on page \pageref{reporting-criteria}, should be relayed to the WFO unless NWS employees have requested that different reporting criteria be used for a particular event (for example, only hail over 1/2'').

\subsection{Send Urgent Reports by Phone}

The NWS has requested that urgent reports –-- such as tornadoes and funnel clouds –-- be called in by telephone in addition to the electronic notification via RMS.  This phone call must be made right away, even if RMS has not yet systematically released the report to NWS.

\subsection{Relaying via Radio}

In the event of an Internet outage or other difficulty in our normal reporting and communication systems, Spotter reports that meet \nameref{reporting-criteria} need to be relayed to the WFO by radio.  The most efficient way to do this is typically by APRS or, if Internet access is available at Net Control, through Winlink.  If there is a total computer outage at the radio desk, or if one party involved in the relay of messages does not have access to packet, Winlink, or another suitable digital messaging platform, a voice relay system should be established.

This voice relay system could consist of the radio desk directly monitoring the local nets, or having an additional Net Controller serve as a relay passing critical messages from local nets to the WFO via a single HF frequency or other communication channel.

In any instance of a difficulty in achieving routine communication, the Responder(s) operating the radio desk are charged with the responsibility of establishing a process for passing traffic in and out of the WFO based on the current communication situation.

%%%%%%%%%%%%%%%%%%%%%%%%%%%%%%%%%%%%%%%%%%%%%%%%%%%%%%%%%%%%%%%%%%%%%%%%

\section{Dealing With the Unexpected}

\subsection{Frequency Conflicts}

In the event a local SKYWARN subnet conflicts with another agency's use of the repeater, every effort should be made to allow SKYWARN to operate on the Primary SKYWARN Repeater.  Under no circumstances should an NCO, Area Manager, or other member of the SKYWARN team engage in any argument or other sort of debate on the air about precedence, who was there first, etc. 

If another net is in progress, establish contact with Net Control and advise them of the SKYWARN activation, and let them know you are the SKYWARN Net Control Station and are on frequency.  Suggest that they continue to run their net and direct traffic for SKYWARN to you. 

In some cases, the other NCO will yield to SKYWARN and either terminate their operations or move them to another frequency.  In other cases, the other NCO will continue running their net and will accept check-ins from stations with traffic for SKYWARN, requesting they "go direct" with you to relay the report.  Or, if the other net is generating relatively little traffic, both nets may coexist with two separate Net Controllers. 

If the other net will not cooperate, check to see if the Backup SKYWARN Repeater is in use. If it is not, politely thank the other NCO and ask them to please periodically announce that SKYWARN is active on the Backup repeater and provide the frequency. 

A brief report of the details should be passed up the SKYWARN chain to the Amateur Radio Coordinator, complete with the details of date, time, frequency, and the specifics of the conflicting operation (net controller name/call sign, agency name, etc).  This will allow SKYWARN to work out an agreement for better coordination with the other agency in the future.

\subsection{Malicious Interference}

Unfortunately it should be expected that a local subnet will at some point in time experience jamming or other intentional interference.  All members of the SKYWARN team are expected to handle these situations professionally and in a manner that does not directly address the interference over the air. 

SKYWARN must make every effort to operate on its Primary Repeaters at all times.  If a station is intentionally causing interference to our operations, moving to another Repeater is pointless; the jammer can change frequencies just as easily as we can, so rather than thwart the offending station's efforts, it simply causes additional unnecessary confusion to Spotters needing to call in reports. 

Since SKYWARN policy is that a minimum of two Net Control Operators should be on the air in each Operating Area at any point in time during every activation, the second NCO should automatically switch to the Backup SKYWARN Repeater and monitor for reports, periodically giving out his or her call sign.  The main NCO (on the Primary SKYWARN Repeater) should make it a point to periodically remind stations of the Backup SKYWARN repeaters, but should remain on frequency. 

It is assumed, of course, that the jammer is only investing enough equipment and effort into causing interference with one frequency at a time.  In cases of more severe interference, enlist the help of a Coordinator right away.  All instances of intentional interference should be reported to SKYWARN Leadership at the conclusion of the activation. 

Again, at no time should the offending station receive any recognition, acknowledgment, or other affirmation of his or her actions on the air!

\subsection{NCO or WFO Under Direct Threat}

If a local Net Control Operator comes under a direct threat, the threatened station is to cease operation immediately.  A simple announcement such as ``I'm under a tornado warning, gotta go!'' is sufficient. Drop the microphone and take cover immediately.

\orangebox{In Case of Emergency}{
When under a direct threat of severe weather, Net Control should not be concerned with taking time to hand off the net to someone else; there should be a second Net Control Operator on frequency already.  If not, someone else (NCO or not) will surely take over the net, or SKYWARN simply goes off the air for a few minutes.\\
\\
\textbf{Do not jeopardize your safety for SKYWARN!}
}
\bigskip

If NWS Wakefield is under a direct threat, such as a Tornado Warning, all operations at the WFO may temporarily cease.  Control of the Wakefield CWA is passed to NWS Newport/Morehead City, NC (WX4MHX).  All SKYWARN personnel in the WFO are to report immediately to the designated Storm Shelter Area, which is the break room adjacent to the Operations area.  NWS Staff will likely place volunteers' safety over their own; cooperate, do not delay... immediately go off the air and take shelter until directed otherwise! 

\orangebox{Good to Know}{
If operations at NWS Wakefield are turned over to NWS Newport/Morehead City for any reason, that office can be reached at (800)679-3373 or (252)223-5122.}

Net Control Operators forced off the air due to storm damage or another emergency should notify any Area Manager or other member of the SKYWARN Leadership team as soon as possible once the threat has passed.

\subsection{Widespread Loss of SKYWARN Repeaters}

In the event of a widespread loss of access to designated SKYWARN Repeaters (for example, if all designated SKYWARN Repeaters in a given Operating Area are off the air) all available Net Control Operators should monitor the designated SKYWARN FM Simplex Frequency for their Area.  The idea is to blanket as much of the area as possible with NCO coverage.

Reports may also be collected via the Wakefield SKYWARN HF Frequencies, which can be used to relay reports to the SKYWARN Desk in the event of a major communications emergency.

\subsection{Telephone Outage in Areas Unreachable by WFO via VHF}

If a severe weather event or other emergency disables telephone service between NWS Wakefield and one or more Operating Areas, and the impacted Area(s) repeaters are unreachable from the WFO, the designated SKYWARN HF Frequencies are to be used to relay reports. 

If the local repeater is still operational, an NCO with simultaneous HF and VHF access can relay reports from the local net to the WFO across the HF link. 

Another alternative is to use APRS messaging.  It may be necessary to utilize a wider path (ie, \verb|WIDE1-2,WIDE2-3|) to establish the initial contact.  Observing the actual path used (such as \verb|WT0MM-1,W4RAT-5|) will eliminate the need for such a wide broadcast path on subsequent messages and should be used after the initial contact is made.  Remember that IGATE's in the impacted area will likely be inoperable.

%%%%%%%%%%%%%%%%%%%%%%%%%%%%%%%%%%%%%%%%%%%%%%%%%%%%%%%%%%%%%%%%%%%%%%%%

\section{Handling Fraudulent or Suspect Reports}

\subsection{Identifying Suspicious Reports}

A \emph{suspicious report} is any report which does not appear to fit the current weather situation or which otherwise just doesn't ``feel'' right.  Examples might include a lone report of a tornado on the ground where no rotation is present on radar, or an abnormally high snowfall total such as seven inches where everyone else in town is reporting four inches.

\orangebox{Be Careful!}{
Suspicious reports may be sent to us with either good or bad intentions.  The Spotter may be misidentifying a low-hanging cloud structure or using poor snowfall measurement techniques.  While very rare, some Spotters do get a kick out of calling in false or exaggerated reports.\\
\\
It is up to the Net Controller to tell the difference between simply a bad report and one that's malicious, and Net Control must be able to handle each situation appropriately.}

\subsection{Probing for Clarity}

As a part of the core Net Control training curriculum, Net Controllers should be trained in asking probing questions to better develop a suspect or incomplete report.  Many times the suspicions surrounding an unexpected report can be alleviated by asking questions that prompt the Spotter to take a closer look at the situation and provide more details about what they are seeing.  Asking the Spotter to describe the shape and movement of clouds, or asking about where and how snowfall measurements were taken, and then providing some brief guidance to the Spotter not only serves us immediately by getting a better quality report right now, it also improves the Spotter's reporting capabilities in the long-term.

It is extremely important that the Spotter not be ``challenged'' or ridiculed on the air, and that any questions asked are done so in such a way as to not convey any inherent disbelief or suspicion about the report. Questions from Net Control should be tactful and courteous in nature.

\subsection{Logging and Relaying Suspect Reports}

RMS includes a ``flag'' feature that allows Net Control to easily identify a suspicious report.  The Comments field should be used to include any remarks about the nature of the suspicion, if necessary.  The Flag feature provides a quick way to alert NWS employees and other SKYWARN team members that we have reason to believe the report is suspicious and it should be handled with care.

If it is necessary to relay a flagged report over the air, the only permissible verbiage which may be used is to simply say ``flagged'' when giving the report.  The report should not be called fraudulent, suspicious, fake, etc, and no on-air inquiries should be made or elaboration given.

\orangebox{Remember!}{Not all suspicious reports are intentional.  Net Control Operators must be very careful in handling these reports so Spotters are not accused of wrongdoing over the air.}

%%%%%%%%%%%%%%%%%%%%%%%%%%%%%%%%%%%%%%%%%%%%%%%%%%%%%%%%%%%%%%%%%%%%%%%%
%%%%%%%%%%%%%%%%%%%%%%%%%%%%%%%%%%%%%%%%%%%%%%%%%%%%%%%%%%%%%%%%%%%%%%%%
%%%%%%%%%%%%%%%%%%%%%%%%%%%%%%%%%%%%%%%%%%%%%%%%%%%%%%%%%%%%%%%%%%%%%%%%
%%%%%%%%%%%%%%%%%%%%%%%%%%%%%%%%%%%%%%%%%%%%%%%%%%%%%%%%%%%%%%%%%%%%%%%%
%%%%%%%%%%%%%%%%%%%%%%%%%%%%%%%%%%%%%%%%%%%%%%%%%%%%%%%%%%%%%%%%%%%%%%%%
%%%%%%%%%%%%%%%%%%%%%%%%%%%%%%%%%%%%%%%%%%%%%%%%%%%%%%%%%%%%%%%%%%%%%%%%

\chapter{Radio Desk Operations}

%%%%%%%%%%%%%%%%%%%%%%%%%%%%%%%%%%%%%%%%%%%%%%%%%%%%%%%%%%%%%%%%%%%%%%%%

\section{Decision to Activate}

The WFO will be staffed with one or more amateur radio volunteers whenever doing so would provide an operational advantage for the ham radio team or NWS employees.  Weather events which are expected to last only a couple of hours or which will impact only a portion of the CWA generally do not warrant the activation of the SKYWARN Radio Desk.

The Amateur Radio Coordinator is responsible for working with the NWS employees to make a decision on whether and when to staff the Radio Desk.

%%%%%%%%%%%%%%%%%%%%%%%%%%%%%%%%%%%%%%%%%%%%%%%%%%%%%%%%%%%%%%%%%%%%%%%%

\section{Staffing Considerations}

Activation of the SKYWARN Radio Desk for short-duration and low-probability events will only serve to wear down the team of Responders.  Travel time, activation duration, and frequency of activations must be taken into consideration, along with the size of the Responder crew and their apparent willingness to come to the WFO for any given activation.

Travel conditions must also be taken into consideration.  Safety of SKYWARN volunteers takes precedence over all other considerations.  Hazardous travel conditions before and after the activation –-- including the likelihood of being stranded at the WFO for an extended period of time –-- must be assessed before deciding to send volunteers to the WFO.

NWS employees will respect any request to withhold staffing of the SKYWARN Radio Desk based on volunteer safety considerations.

%%%%%%%%%%%%%%%%%%%%%%%%%%%%%%%%%%%%%%%%%%%%%%%%%%%%%%%%%%%%%%%%%%%%%%%%

\section{Service Level Commitment}

The SKYWARN Radio Desk Service Level Commitment, which specifies the hours during which amateur radio support will be provided at the WFO, is specified in the section called ``\nameref{radio-desk-slc}'' on page \pageref{radio-desk-slc}.

%%%%%%%%%%%%%%%%%%%%%%%%%%%%%%%%%%%%%%%%%%%%%%%%%%%%%%%%%%%%%%%%%%%%%%%%

\section{Guests}

Guests should not be brought along to the Forecast Office during activations.  The NWS office is not the place for guests or sightseers during SKYWARN activations.  Due to United States Department of Homeland Security and Department of Commerce regulations, all visitors including amateur radio operators must present identification and state their reason for entry into the government facility.  The NWS would be pleased to give a tour of the office during quiet weather and upon prior arrangement. This can be arranged by calling the office; contact information can be found on the WFO web site at \href{http://www.nws.noaa.gov/er/akq}{http://www.nws.noaa.gov/er/akq}.

Special advance arrangements can be made for visitation of emergency management partners and certain SKYWARN team members wishing to observe SKYWARN and/or NWS operations during an activation. The Amateur Radio Coordinator must make these arrangements with the Meteorologist in Charge (MIC) or Warning Coordination Meteorologist (WCM).

%%%%%%%%%%%%%%%%%%%%%%%%%%%%%%%%%%%%%%%%%%%%%%%%%%%%%%%%%%%%%%%%%%%%%%%%

\section{NWS Office Operating Conditions}

When SKYWARN is activated, NWS personnel are usually operating under high tension in a critical weather mode. Forecasters and other staff members are under intense pressure.  This means: 

\begin{enumerate}
\item Any distractions or interruptions of NWS or SKYWARN operations may mean the loss of life or property. 
\item Sensitive information such as reports of severe damage, deaths, or injuries may be openly discussed and such information should not be repeated by SKYWARN volunteers on the air or outside the NWS. 
\item No more than three SKYWARN volunteers should be in the Operations area at any time. All other volunteers in the building should be staged in the Conference Room or other designated location.  The Operations area is very busy during severe weather and traffic through this area should be kept to an absolute minimum. Ideally, one Responder should attempt to position himself as close to the radar/warning position as possible without getting in the way of forecasters.  The other one or two Responders should remain at the SKYWARN Radio Desk.
\end{enumerate}

%%%%%%%%%%%%%%%%%%%%%%%%%%%%%%%%%%%%%%%%%%%%%%%%%%%%%%%%%%%%%%%%%%%%%%%%

\section{Volunteering for Responder Duty}

Responders should not go directly to the NWS office or call the NWS office at the first sign of severe weather. To be an effective and well-coordinated system, we must follow protocol: 

\begin{enumerate}
\item NWS and/or the Amateur Radio Coordinator determines the need for SKYWARN activation and notifies the appropriate personnel. 
\item The Amateur Radio Coordinator contacts Responders to man the NWS station and will notify the Area Manager(s) in the Operating Area(s) where severe weather is expected, who will in turn contact the individual Net Control Operators to begin preparations for the local nets. 
\item Responders may contact the Amateur Radio Coordinator to inform him of their availability during a quiet time in net operations.  Do not be insulted if your services are not needed at that time. As the situation evolves, staffing needs may also change.
\end{enumerate}

\orangebox{Do Not Self-Deploy}{SKYWARN Team Members should never go directly to the National Weather Service Office without official activation instructions.}

%%%%%%%%%%%%%%%%%%%%%%%%%%%%%%%%%%%%%%%%%%%%%%%%%%%%%%%%%%%%%%%%%%%%%%%%

\section{Interaction with NWS Employees}

The forecaster who briefs the first Responder upon arrival at the NWS will likely be the contact person until the NWS shift changes.  Please follow your instincts on how and when to pass information to the forecasters. If the information is critical, bring this information to the forecaster's attention immediately.  Otherwise, you will need to gauge the situation as to whether the information is important enough to bring to the forecaster's immediate attention or if it can wait a few minutes.  It is a delicate balance to make this critical part of net operation successful and it must be handled with discretion, tact and diplomacy by the Responder.

One Responder should maintain contact with forecasters to a) provide the latest reports and to b) keep up with the latest statements, warnings, and areas of concern.

%%%%%%%%%%%%%%%%%%%%%%%%%%%%%%%%%%%%%%%%%%%%%%%%%%%%%%%%%%%%%%%%%%%%%%%%

\section{Access to the WFO}

SKYWARN Officials and Responders operating from NWS Wakefield should park in the front parking lot along the side closest to Route 460.  Use the front entrance to gain entry to the NWS office. This door is locked at all times.  To gain entry, push the button on the left wall and a buzzer will sound inside. 

SKYWARN Officials and Responders must present their official SKYWARN ID badges or valid government-issued photo identification to gain entry to the NWS Wakefield facility.  The National Weather Service has the right to refuse entry to persons not providing sufficient identification and has been asked to restrict access to the SKYWARN Radio Desk if sufficient identification cannot be provided.

%%%%%%%%%%%%%%%%%%%%%%%%%%%%%%%%%%%%%%%%%%%%%%%%%%%%%%%%%%%%%%%%%%%%%%%%

\section{Personal Items}

Responders are required to provide their own food and beverages in a sufficient quantity to make it through the activation. Survival basics and hygiene items are also to be furnished by the individual Responder.

Responders are required to keep the amount of personal items brought into the WFO to a bare minimum.  In general, clothing and other hygiene items should be kept in a backpack in the supply closet at the radar end of the hall.   A refrigerator, freezer, microwave, oven, and coffee maker are available for SKYWARN use in the break room.   Please be mindful of the amount of food brought into the building.

A sleeping bag and/or mat, pillow, and change of clothes is a good idea if there is any chance the activation may run into the late evening or overnight hours.

The maximum amount of personal belongings permitted inside the WFO at any point in time is as follows (per person):
\begin{itemize}
\item 1 standard backpack-size bag or container of clothing, food, and/or personal hygiene items.
\item 1 standard paper/2 standard plastic grocery bags of food/snacks.
\end{itemize}

In addition, during an overnight stay, the following items may be brought in to the WFO only for the period of time spent sleeping inside the building, and must be promptly removed from the building upon waking up in the morning:
\begin{itemize}
\item 1 sleeping bag and/or sleeping mat
\item 1 pillow
\end{itemize}

Exceptions are of course permitted for equipment and supplies that are medically necessary.

%%%%%%%%%%%%%%%%%%%%%%%%%%%%%%%%%%%%%%%%%%%%%%%%%%%%%%%%%%%%%%%%%%%%%%%%

\section{Briefing Upon Arrival}

Upon arrival at the NWS, the Responder should immediately identify himself to a forecaster or other staff member as a SKYWARN amateur radio operator and ask a forecaster for a briefing on the severe weather situation, attempting to get the following information:

\begin{enumerate}
\item Where storms are located and in which direction they are moving. 
\item Characteristics and history of the storm(s) (hail, damaging winds, tornadoes, etc.) 
\item What geographic locations are of primary concern to the forecasters. 
\item The latest statements, watches, and warnings to be read over the net.
\end{enumerate}

If another Responder is already on site, he or she should provide this information to incoming volunteers if time and operating conditions permit.

%%%%%%%%%%%%%%%%%%%%%%%%%%%%%%%%%%%%%%%%%%%%%%%%%%%%%%%%%%%%%%%%%%%%%%%%

\section{Security Policies}
SKYWARN personnel must abide by all posted policies regarding the security of the Forecast Office, including regulations regarding building access, visitor sign-in and identification, entry and exit policies.  Failure to do so may result in immediate and permanent removal from the office.

%%%%%%%%%%%%%%%%%%%%%%%%%%%%%%%%%%%%%%%%%%%%%%%%%%%%%%%%%%%%%%%%%%%%%%%%
%%%%%%%%%%%%%%%%%%%%%%%%%%%%%%%%%%%%%%%%%%%%%%%%%%%%%%%%%%%%%%%%%%%%%%%%
%%%%%%%%%%%%%%%%%%%%%%%%%%%%%%%%%%%%%%%%%%%%%%%%%%%%%%%%%%%%%%%%%%%%%%%%
%%%%%%%%%%%%%%%%%%%%%%%%%%%%%%%%%%%%%%%%%%%%%%%%%%%%%%%%%%%%%%%%%%%%%%%%
%%%%%%%%%%%%%%%%%%%%%%%%%%%%%%%%%%%%%%%%%%%%%%%%%%%%%%%%%%%%%%%%%%%%%%%%
%%%%%%%%%%%%%%%%%%%%%%%%%%%%%%%%%%%%%%%%%%%%%%%%%%%%%%%%%%%%%%%%%%%%%%%%

\chapter{WX4AKQ Station Regulations}

%%%%%%%%%%%%%%%%%%%%%%%%%%%%%%%%%%%%%%%%%%%%%%%%%%%%%%%%%%%%%%%%%%%%%%%%

\section{Call Sign Usage}\label{wx4akq-usage}

The WX4AKQ call sign is to be used for all SKYWARN operations taking place from the SKYWARN amateur radio station located inside NWS Wakefield.  Net Control Stations operating the local nets from outside the NWS office should identify with their personal call sign. 

The WX4AKQ call sign may also be used for SKYWARN-related special events, such as ARRL Field Day, SKYWARN Appreciation Day, and other events as authorized by the license Trustee. 

The WX4AKQ call sign may also be used on any SKYWARN-owned and operated repeater or relay station, including APRS and packet nodes, as authorized by the Trustee.

The Responder physically located at the National Weather Service office, as well as Net Control Operators running the local nets, may use the tactical call sign ``SKYWARN Net Control'' to identify.  FCC rules apply, so the operator must also identify himself using the WX4AKQ call once every ten minutes and at the end of each contact.  Additional information on station identification can be found in the section called ``\nameref{nco-station-id}'' on page \pageref{nco-station-id}.

%%%%%%%%%%%%%%%%%%%%%%%%%%%%%%%%%%%%%%%%%%%%%%%%%%%%%%%%%%%%%%%%%%%%%%%%

\section{Operating Privileges, Control Operators}

The use of the WX4AKQ club call sign does not provide any operating privileges.  Operating privileges (frequencies, emissions, etc) are determined by the privileges of the Responder's personal license class, or those of the station control operator(s). 

As a matter of our policy, the Responder(s) holding the highest class of amateur license are considered the control operators for the station.  For example, if one Technician and two General licensees are present, the two General licensees share the responsibility as control operator, and the General Class license privileges may be used by the Technician operator while under the direct supervision of one or both of the General operators.  As an additional example, if one Technician, one General, and one Extra Class operator are present, the Extra Class operator is considered the control operator and is responsible for all station activities. 

In accordance with FCC regulations, at no time may an amateur operate outside his or her license privileges without a control operator present, and at that point operation is limited to the privileges of the control operator, and all such transmissions shall be at the direction and under the supervision of the control operator.

%%%%%%%%%%%%%%%%%%%%%%%%%%%%%%%%%%%%%%%%%%%%%%%%%%%%%%%%%%%%%%%%%%%%%%%%

\section{Radio Desk Operator Log}

The WX4AKQ Operator Sign-In Sheets will be used to keep track of station operators and control operators and are maintained in an orange folder at the SKYWARN desk.  All persons operating the SKYWARN amateur radio station are required to sign in and out with the requested information, which includes date and time of arrival and departure and his/her personal call sign.  Notify the Amateur Radio Coordinator of any errors or omissions in this log as soon as possible.

%%%%%%%%%%%%%%%%%%%%%%%%%%%%%%%%%%%%%%%%%%%%%%%%%%%%%%%%%%%%%%%%%%%%%%%%

\section{WX4AKQ License Trustee}
The Amateur Radio Coordinator shall be listed as the Trustee on the WX4AKQ station license.  If there is a change in leadership, the incoming Amateur Radio Coordinator and Warning Coordination Meteorologist are responsible for updating the FCC records with the name and call sign of the new Trustee.

%%%%%%%%%%%%%%%%%%%%%%%%%%%%%%%%%%%%%%%%%%%%%%%%%%%%%%%%%%%%%%%%%%%%%%%%

\section{Modification or Renewal of WX4AKQ License}

From time to time it will be necessary to make changes to the WX4AKQ license, such as a Trustee change, address change, or renewal.  Because WX4AKQ is a vanity club call sign, a Club Station Call Sign Administrator (CSCSA) such as W5YI-VEC must be used to process these changes. While administrative changes are usually performed free of charge, a standard vanity call sign renewal fees apply to license renewals. 

In general, when making changes to a club call sign, supporting documents in the form of club constitutions or other similar material must be submitted.  Because Wakefield SKYWARN is not organized as a club and is instead serving the National Weather Service, a special procedure is in place to handle these changes. 

The Trustee (or incoming Amateur Radio Coordinator if changing Trustees) must sign off on the modification paperwork.  The Warning Coordination Meteorologist will prepare a brief statement on National Weather Service letterhead stating that the call sign is used for purposes of SKYWARN.  The letter must also authorize you, by name and call sign, to become the Trustee or make the other administrative changes. 

This letter must be submitted along with the application paperwork to the CSCSA, along with any required processing fee(s).  Processing fee(s) are generally paid by the National Weather Service Forecast Office directly.

%%%%%%%%%%%%%%%%%%%%%%%%%%%%%%%%%%%%%%%%%%%%%%%%%%%%%%%%%%%%%%%%%%%%%%%%

\section{WX4AKQ Station Equipment}

The SKYWARN Amateur Radio Station consists of five radios.  There are two Yaesu FT-2800M 2-meter FM transceivers and two Kenwood TM-441A 70-centimeter FM transceivers.  There is also a Kenwood TS-570D HF transceiver for monitoring and communication with the National Hurricane Center and other HF nets. Instructions on the operation of all equipment in the SKYWARN amateur radio station can be found in the bottom right-hand drawer of the amateur radio. 

Two Kenwood TM-241A 2-meter FM transceivers are kept in secure on-site storage for use in the event of a failure in the primary Yaesu radios, or for use in future VHF data projects.  Certain NWS staff members have a key to the on-site storage facility.  If there is a need to swap a radio, notify an NWS staff member who will assist in gaining access to the equipment. 

One of the Yaesu FT-2800M transceivers is used full-time for APRS applications.  If there is a need for a second radio for voice communication, the TNC can be easily disconnected from the radio's microphone and speaker jacks.  The hand microphone for the FT-2800M is be kept hanging directly next to this radio for voice use at a moment's notice. 

Responders are encouraged to become familiar with the operation of these radios prior to responding to the NWS office for an activation.  Where possible, electronic versions of the equipment manuals are kept on the SKYWARN computer and on the SKYWARN Amateur Radio Support Team web site.

%%%%%%%%%%%%%%%%%%%%%%%%%%%%%%%%%%%%%%%%%%%%%%%%%%%%%%%%%%%%%%%%%%%%%%%%

\section{WX4AKQ Equipment Maintenance}\label{wx4akq-maint}

All volunteers using the SKYWARN amateur radio station have a responsibility for assisting with maintenance of station equipment.  This responsibility is best carried out by notifying the Amateur Radio Coordinator of any problems with the equipment.  The Amateur Radio Coordinator will pursue repair or replacement of equipment as necessary. 

The Amateur Radio Coordinator may designate a Tech Team consisting of qualified individuals willing to assist with the periodic maintenance of the station equipment, as needed.

%%%%%%%%%%%%%%%%%%%%%%%%%%%%%%%%%%%%%%%%%%%%%%%%%%%%%%%%%%%%%%%%%%%%%%%%

\section{Ownership of WX4AKQ Equipment}

All equipment permanently installed at National Weather Service is the property of the United States Department of Commerce, unless otherwise clearly and permanently marked otherwise. 

SKYWARN radio equipment shall not be removed from the WFO under any circumstances without prior authorization from both the Amateur Radio Coordinator and either the Warning Coordination Meteorologist or Meteorologist-in-Charge.  Since this is United States Government property, serious consequences can result from tampering or unauthorized removal.

%%%%%%%%%%%%%%%%%%%%%%%%%%%%%%%%%%%%%%%%%%%%%%%%%%%%%%%%%%%%%%%%%%%%%%%%

\section{Equipment Loans}

Occasionally it may be necessary for a Team member or other individual to supply radio equipment or related accessories on a temporary basis, for example, in the event of the failure of a radio, antenna, power supply, microphone, or feedline at the SKYWARN amateur radio station.  Neither the SKYWARN Amateur Radio Support Team nor its leadership nor the National Weather Service shall bear any liability for loss or damages to such equipment or accessories. 

%%%%%%%%%%%%%%%%%%%%%%%%%%%%%%%%%%%%%%%%%%%%%%%%%%%%%%%%%%%%%%%%%%%%%%%%

\section{Equipment Donations}

From time to time a member, amateur, club, or organization may wish to donate equipment to the SKYWARN Amateur Radio Support Team.  These donations shall be handled by the Amateur Radio Coordinator and a list of all donated equipment will be maintained by the Amateur Radio Coordinator. Updated copies of this list will be provided to the SKYWARN Program Manager on a regular basis and posted at the SKYWARN amateur radio station to account for the ownership of this equipment.

%%%%%%%%%%%%%%%%%%%%%%%%%%%%%%%%%%%%%%%%%%%%%%%%%%%%%%%%%%%%%%%%%%%%%%%%
%%%%%%%%%%%%%%%%%%%%%%%%%%%%%%%%%%%%%%%%%%%%%%%%%%%%%%%%%%%%%%%%%%%%%%%%
%%%%%%%%%%%%%%%%%%%%%%%%%%%%%%%%%%%%%%%%%%%%%%%%%%%%%%%%%%%%%%%%%%%%%%%%
%%%%%%%%%%%%%%%%%%%%%%%%%%%%%%%%%%%%%%%%%%%%%%%%%%%%%%%%%%%%%%%%%%%%%%%%
%%%%%%%%%%%%%%%%%%%%%%%%%%%%%%%%%%%%%%%%%%%%%%%%%%%%%%%%%%%%%%%%%%%%%%%%
%%%%%%%%%%%%%%%%%%%%%%%%%%%%%%%%%%%%%%%%%%%%%%%%%%%%%%%%%%%%%%%%%%%%%%%%

\chapter{Interoperability Plan}

Our ability to function alongside other EMCOMM organizations requires advance planning and careful coordination to ensure efficient sharing of limited airspace.  In addition, the team must adhere to established communication protocols to ensure effective communication with other agencies.

This Interoperability Plan (``Interop Plan'') describes our plans for coordinating our activities with other organizations and communicating with other agencies.  This plan uses a two-part approach depending on the type of activation:  ``routine event'' and ``major event.''

%%%%%%%%%%%%%%%%%%%%%%%%%%%%%%%%%%%%%%%%%%%%%%%%%%%%%%%%%%%%%%%%%%%%%%%%

\section{Activation Types Defined}

A ``routine event'' activation is any activation meeting these criteria:

\begin{itemize}
\item Local nets are activated in one or more Operating Areas.
\item The SKYWARN Radio Desk may or may not be activated.
\item State Emergency Operations Centers (EOC's) are not activated.
\item Local EOC's may or may not be activated.
\item Event duration is generally 6 hours or less.
\end{itemize}

Examples include most summertime severe weather, short periods of freezing rain or snow which cause brief disruptions to travel, localized flooding, etc.

A ``major event'' is an activation meeting two or more of these criteria:

\begin{itemize}
\item Local nets active in two or more Operating Areas.
\item SKYWARN Radio Desk is activated.
\item At least one state EOC is activated.
\item Event duration is generally 12 hours or more.
\end{itemize}

Examples include tropical storms, hurricanes, major winter storms, and large-scale severe weather outbreaks such as large squall lines and derechos.

%%%%%%%%%%%%%%%%%%%%%%%%%%%%%%%%%%%%%%%%%%%%%%%%%%%%%%%%%%%%%%%%%%%%%%%%

\section{Routine Event Interop Plan}

During most routine events, SKYWARN is the only EMCOMM group on the air.  ARES teams may be on standby or in the early stages of activating, or, in perhaps a rare case, could have nets on the air.  The greatest concern for potential on-air conflicts would be local club nets, traffic nets, etc.

For these events, coordination with other nets is best handled by the SKYWARN net controller or the Area Manager.  It is not practical to hold a club net alongside a SKYWARN net, so most club nets will gladly yield to SKYWARN activity by delaying or canceling the net or by moving it to an alternate frequency.  SKYWARN generally takes precedence in these situations, but if we are asked to move to another frequency, we will honor that request.  The Area Manager or Amateur Radio Coordinator should be engaged by Net Control to work out any disagreements that might arise.

Interagency communication is typically not a concern for these routine events.  While several localities might experience a widespread loss of utilities such as power or telephone, communications are not often impacted at such a time and in such a manner as to require an alternate communication path be established between the National Weather Service and one of those localities.  In the unlikely event such a need arises, the SKYWARN Radio Desk can be activated to provide that link.

Standard SKYWARN frequency plans will typically be used.  The team will maintain both ``plain language'' and ICS 205 formatted standard frequency plans for each Operating Area and make them available on the team web site.  Frequency changes necessitated by cooperation with other repeater users may result in temporary changes to alternate, unpublished frequencies.  In this instance, the user of the standard SKYWARN repeater will be asked to help direct SKYWARN traffic to the new frequency.

HF frequencies may be used to collect reports from outlying areas not served by VHF/UHF SKYWARN repeaters.  These nets would be operated by regular SKYWARN Net Control Operators.

%%%%%%%%%%%%%%%%%%%%%%%%%%%%%%%%%%%%%%%%%%%%%%%%%%%%%%%%%%%%%%%%%%%%%%%%

\section{Major Event Interop Plan}

Major events involve significant severe weather over a large portion of the County Warning Area.  State and local EOC's are likely active and this plan includes establishing a communications link between the National Weather Service and the state EOC's.

In all instances, if the Old Dominion Emergency Network (ODEN) is active, the SKYWARN Radio Desk will check in to that net and will maintain a presence there to send and receive formal message traffic from served agencies.  Under no circumstances will Spotter reports be handled across this net.

If ODEN is not active but the North Carolina state EOC is active, WX4AKQ will establish an HF communications link with that EOC via any statewide emergency net or other designated frequency.  If North Carolina's EOC is not active but Maryland's EOC is active, an HF link will be established with Maryland's EOC via any statewide emergency net or other designated frequency.

If no state EOC is active, the ODEN frequency will be monitored for traffic bound for NWS Wakefield.

\emph{NOTE: Consult the latest WX4AKQ Standard Communications Plan ICS 205 documents for the current frequencies assigned to Virginia, North Carolina, and Maryland emergency nets.}

Communication with local EOC's will be coordinated through the appropriate statewide emergency net whenever voice communication is required.  In all instances, the preferred method by which formal message traffic should be passed is Winlink, and all messages shall be in standard ICS 213 format.

Prior to the activation of WX4AKQ, an ICS Compliant communication plan (ICS 205) will be prepared and transmitted to all team members and emergency management partners, and will be posted on the team web site.

During major events, there is a high likelihood that SKYWARN will be sharing VHF/UHF airspace with other nets.  It is very easy to run two emergency nets side-by-side on one frequency:  the two Net Controls alternate calls for check-ins and take turns handling traffic for their respective nets.  Stations breaking in to a net should indicate which of the nets they need to communicate with.  Should a frequency become too congested trying to share airspace, one net will need to move to another frequency.  If SKYWARN's alternate/backup repeater is available, SKYWARN will move to that frequency.  Whichever net stays behind on the original frequency will be asked to assist in directing traffic to the other net.

The Amateur Radio Coordinator is responsible for working out communication conflicts during major event activations and should be consulted immediately if any on-air conflict cannot be promptly resolved to the satisfaction of all parties involved.  Historically, good sense and common courtesy have prevailed and there have been no major issues.  We expect this to continue.

%%%%%%%%%%%%%%%%%%%%%%%%%%%%%%%%%%%%%%%%%%%%%%%%%%%%%%%%%%%%%%%%%%%%%%%%
%%%%%%%%%%%%%%%%%%%%%%%%%%%%%%%%%%%%%%%%%%%%%%%%%%%%%%%%%%%%%%%%%%%%%%%%
%%%%%%%%%%%%%%%%%%%%%%%%%%%%%%%%%%%%%%%%%%%%%%%%%%%%%%%%%%%%%%%%%%%%%%%%
%%%%%%%%%%%%%%%%%%%%%%%%%%%%%%%%%%%%%%%%%%%%%%%%%%%%%%%%%%%%%%%%%%%%%%%%
%%%%%%%%%%%%%%%%%%%%%%%%%%%%%%%%%%%%%%%%%%%%%%%%%%%%%%%%%%%%%%%%%%%%%%%%
%%%%%%%%%%%%%%%%%%%%%%%%%%%%%%%%%%%%%%%%%%%%%%%%%%%%%%%%%%%%%%%%%%%%%%%%

\chapter{IT Systems Policies}

%%%%%%%%%%%%%%%%%%%%%%%%%%%%%%%%%%%%%%%%%%%%%%%%%%%%%%%%%%%%%%%%%%%%%%%%

\section{Information Technology Systems Overview}

Information Technology (IT) systems refers to all electronic communications systems utilized by the SKYWARN program.  The majority of these systems are operated in-house by the team or are contracted out to a third party, but in some instances these systems may be operated by the National Oceanic and Atmospheric Administration, the United States Department of Commerce, or another government agency.

The policies defined in the following sections of this manual define the policies that relate to the use of SKYWARN operated computer systems.  Other systems and services have additional and separate policies which are in effect.  In all instances, the use of any SKYWARN computer system by any SKYWARN team member, partner, NWS employee, or other individual or organization is subject to the terms set forth herein.

%%%%%%%%%%%%%%%%%%%%%%%%%%%%%%%%%%%%%%%%%%%%%%%%%%%%%%%%%%%%%%%%%%%%%%%%

\section{Acceptable Use Policy}

All SKYWARN IT systems are subject to an Acceptable Use Policy (AUP) as defined herein.

\subsection{Access to Systems}
SKYWARN computer systems are provided for the use of SKYWARN team members, partners, and, in some cases, NWS employees or the general public.   With the exception of systems which are specifically intended for public consumption, access to SKYWARN systems is restricted to each system's intended audience.  Access to these systems is subject to occasional review and is subject to termination at any time, for any reason, or for no reason whatsoever.

\subsection{Authentication}

Most systems are secured with a single-factor authentication mechanism –-- generally a combination of a user identifier and a password.  Users are responsible for maintaining the security of their passwords.  In the event a password may have been compromised, the user must immediately change the password and notify IT Support personnel of the potential compromise.

Some systems may utilize two-factor or multi-factor authentication, which could involve the use of an access card (direct-swipe magnetic stripe card or proximity identification such as RFID or Bluetooth), hardware encryption (USB key, Yubikey) or other physical security measure (fingerprint scanner, facial identification, hardware lock/key).

Upon termination from the team or a change in roles which results in loss of access to one or more SKYWARN systems, all authentication mechanisms controlling access to those systems shall be terminated by the departing team member's supervisor or SKYWARN IT Support personnel.  In the instance of physical authentication, any access cards, keys, or other hardware shall be immediately returned.  Team members may be held financially responsible for unreturned equipment.

\subsection{Permitted Use}

All SKYWARN systems are provided for ``official business'' which includes, but is not necessarily limited to, coordination of SKYWARN activities, performing Net Control duties, collection and communication of Spotter reports, team-related e-mail and other electronic communications, and for training purposes.  Practically any use of SKYWARN systems that relates to SKYWARN nets, reporting, training, communications, and outreach could be considered appropriate uses.

Some team members may have access to certain computer services for personal use, such as personal web space, experimental/hobby programming, etc.  All content published on SKYWARN servers is subject to editorial review by SKYWARN IT Support personnel and may be edited or removed if it contains obscene, abusive, sexually explicit, political, religious, or other content deemed inappropriate within a SKYWARN context.  Personal use of SKYWARN services shall not interfere with SKYWARN operations.

\subsection{Forbidden Use}

The use of any SKYWARN systems for illegal or unethical purposes, or the dissemination of copyrighted, age-restricted, or sexually explicit materials, is expressly forbidden.  In addition, any attempts to undermine or circumvent electronic and/or physical security measures in place on SKYWARN computer systems is prohibited.

\subsection{Logging and Monitoring}

Users should have no expectation of privacy while using any SKYWARN computer system.  All activity is subject to logging, interception, recording, duplication and disclosure as required for the routine maintenance of SKYWARN systems and for investigation and prosecution of unauthorized activities.  This includes e-mail and chat messages received and transmitted through SKYWARN servers.

System log files may include details of the source and target IP addresses of each connection to SKYWARN systems, user authentication information, and all or part of the information transmitted during each session.  These log files are generally retained for a standard period of time, after which they are rotated out of the system; however, as deemed necessary by IT Support, these logs may be retained for a longer period of time.

\subsection{Rights to Stored Data}

All information stored on SKYWARN computer systems may become the property of the team.  Reasonable exceptions are of course made for content published within the confines of personal web space on SKYWARN servers.

Upon termination of access to SKYWARN services, all reasonable efforts will be made to provide the departing team member with a copy of his or her data, such as e-mail, chat logs, and files.  If the team member is leaving due to action stemming from unauthorized use of a SKYWARN computer system, the release of this data may be delayed or otherwise restricted by legal process.

Information stored within SKYWARN computer systems may be released to law enforcement personnel for purposes of investigating suspected unauthorized access or abuse of SKYWARN services, or in response to a valid court order.

\subsection{Unauthorized Use Subject to Prosecution}

Anyone who intentionally or repeatedly engages in prohibited activities may lose access to SKYWARN systems.  Team members may be suspended or removed from the team permanently.  Anyone engaging in forbidden activities is subject to law enforcement action and criminal prosecution.
Content Filtering and Restrictions

Various SKYWARN systems, including VPN and the SKYWARN Responder Desktop, may utilize various technologies that restrict access to objectionable or inappropriate material.  Attempts to circumvent these restrictions are prohibited.  Users wishing to gain access to a blocked site should make a request to IT Support.

\subsection{Display of Acceptable Use Policy Summary}

Where permitted by the technical capabilities of SKYWARN systems, the following Acceptable Use Policy Summary shall be displayed upon initial connection to restricted systems:

\begin{quote}
User Warning:

You have accessed a computer system operated on behalf of a U.S. Government Agency. If you are not an authorized user, please disconnect now.

Unauthorized access or use of this computer system may subject violators to criminal, civil, and/or administrative action.  All information on this computer system may be intercepted, recorded, read, copies, and disclosed by and to authorized personnel for official purposes, including system maintenance and quality control purposes or for criminal investigations.  Access or use of this computer system by any person, whether authorized or unauthorized, constitutes consent to these terms.

This system is for the exclusive use of the Wakefield SKYWARN Amateur Radio Support Team and specific individuals associated with designated partners of the SKYWARN program.  This system is for official SKYWARN operations and training purposes only.

Use of this system is subject to the terms of the SKYWARN IT Systems Policies as defined in the SKYWARN Operations Manual.

In the event that a particular system is unable to display the full text of the summary warning, a text link to a copy of the summary may be provided instead.
\end{quote}

%%%%%%%%%%%%%%%%%%%%%%%%%%%%%%%%%%%%%%%%%%%%%%%%%%%%%%%%%%%%%%%%%%%%%%%%
%%%%%%%%%%%%%%%%%%%%%%%%%%%%%%%%%%%%%%%%%%%%%%%%%%%%%%%%%%%%%%%%%%%%%%%%
%%%%%%%%%%%%%%%%%%%%%%%%%%%%%%%%%%%%%%%%%%%%%%%%%%%%%%%%%%%%%%%%%%%%%%%%
%%%%%%%%%%%%%%%%%%%%%%%%%%%%%%%%%%%%%%%%%%%%%%%%%%%%%%%%%%%%%%%%%%%%%%%%
%%%%%%%%%%%%%%%%%%%%%%%%%%%%%%%%%%%%%%%%%%%%%%%%%%%%%%%%%%%%%%%%%%%%%%%%
%%%%%%%%%%%%%%%%%%%%%%%%%%%%%%%%%%%%%%%%%%%%%%%%%%%%%%%%%%%%%%%%%%%%%%%%

\chapter{System-Specific Policies}

%%%%%%%%%%%%%%%%%%%%%%%%%%%%%%%%%%%%%%%%%%%%%%%%%%%%%%%%%%%%%%%%%%%%%%%%

\section{Situation Awareness Dashboard}

The Situation Awareness Dashboard provides access to critical weather data and various SKYWARN services such as real-time spotter reports, weather alert notifications, activation bulletins, and more.

\subsection{Eligibility}

Access is restricted to individuals meeting the following eligibility requirements:

\textbf{NWS Wakefield SKYWARN Spotters.} Must furnish valid Spotter ID or other confirmation of certification.

\textbf{Neighboring SKYWARN amateur radio teams.} Net Controls and leadership personnel of SKYWARN amateur radio teams serving the Raleigh, Newport/Morehead City, Blacksburg, Sterling, and Mt. Holly offices may register for access.

\textbf{NWS Employees.} Employees of NWS WFO's engaged in forecast operations. Includes MIC, WCM, HMT, forecasters, and interns. Must register with \verb|@noaa.gov| e-mail address.

\textbf{ARES Officials and Appointees.} Registration is open to ARES OESs, ASECs, ADECs, DECs, ECs, AECs, SMs, and ASMs serving any of the 66 counties and independent cities within the Wakefield County Warning Area.

\textbf{Local Emergency Management Officials.} Local emergency management personnel in the 66 counties and independent cities within the Wakefield County Warning Area. Must register with valid local government e-mail address.

\textbf{State Emergency Management Officials.} State-level emergency management officials in Virginia, North Carolina, and Maryland. Must register with valid state government agency e-mail address.

\subsection{Privacy Notice}

The information submitted during the registration process will be used to establish and periodically verify eligibility for access to this system.  Personally identifiable information, including name, address, telephone number, e-mail address, job function, call sign, IP address, or any other personal attributes, will be available to system administrators only.  This information may be revealed to law enforcement personnel in response to a valid court order, or as a part of an investigation into suspected unauthorized access to this or any other Wakefield SKYWARN system.  Otherwise, personally identifiable information will never be shared with any third parties without user consent.

\subsection{E-mail Alerts Disclaimer}
The Situation Awareness Dashboard includes access to an e-mail weather alert service for some users, depending on account type.  This alert service is provided on a best-effort basis and is subject to delays in transmission and other problems associated with Internet-based systems, including server downtime, disruptions to the incoming product feed, e-mail outages, message delivery failures at our end, the user's end, or anywhere in between, programming glitches, detection problems, routing failures, and any of a number of other problems.  The Wakefield SKYWARN Amateur Radio Support Team built this alert system from the ground up and has found it very reliable since it went into operation in 2010.  However, downtime does happen!

Products are only distributed by the NOAA National Telecommunications Gateway once every two minutes. Additional delays may occur if our system is running behind or otherwise having problems.  Do not rely on this service for mission critical applications or for the protection of life or property.  There are a number of commercial services available that are better suited for some users. EMWIN is also available over-the-air via VHF radio, satellite, and other sources.  We are not responsible for any problems resulting from the use of (or any inability to use) this service.

\subsection{Net Log Viewer Disclaimer}

Reports contained in the SKYWARN Net Log Viewer are not official storm reports and are provided only for the convenience of our emergency management partners. Official NWS storm reports which have been ``sanity checked'' are published in the NWS Local Storm Report (LSR) and/or Public Information Statement (PNS). Reports are displayed here once they are relayed to the National Weather Service for processing, but the display of a report on this page does not mean the report has been received or understood by NWS employees.  ``Log Only'' reports are deemed by the Net Controller to not meet \nameref{reporting-criteria} for the current event and were not relayed to the National Weather Service.  Reports contained within this system are confidential and any dissemination should be limited to emergency management personnel on a ``need to know'' basis.

\subsection{Warranty Statement}

This web site and the features contained herein are provided AS-IS, without any warranty or guarantees of functionality, reliability, or fitness for any particular purpose. We strive for 100\% uptime, but periodic system outages do occur, whether a result of technical difficulties, user error, system upgrades, or other planned or unplanned maintenance operations. We will make every effort to notify users of planned system outages as far in advance as possible, but there may be certain instances in which such notification is not possible. Do not depend on this system for mission-critical applications or for the protection of life or property.

%%%%%%%%%%%%%%%%%%%%%%%%%%%%%%%%%%%%%%%%%%%%%%%%%%%%%%%%%%%%%%%%%%%%%%%%
%%%%%%%%%%%%%%%%%%%%%%%%%%%%%%%%%%%%%%%%%%%%%%%%%%%%%%%%%%%%%%%%%%%%%%%%
%%%%%%%%%%%%%%%%%%%%%%%%%%%%%%%%%%%%%%%%%%%%%%%%%%%%%%%%%%%%%%%%%%%%%%%%
%%%%%%%%%%%%%%%%%%%%%%%%%%%%%%%%%%%%%%%%%%%%%%%%%%%%%%%%%%%%%%%%%%%%%%%%
%%%%%%%%%%%%%%%%%%%%%%%%%%%%%%%%%%%%%%%%%%%%%%%%%%%%%%%%%%%%%%%%%%%%%%%%
%%%%%%%%%%%%%%%%%%%%%%%%%%%%%%%%%%%%%%%%%%%%%%%%%%%%%%%%%%%%%%%%%%%%%%%%

\chapter{Miscellaneous Policies}

%%%%%%%%%%%%%%%%%%%%%%%%%%%%%%%%%%%%%%%%%%%%%%%%%%%%%%%%%%%%%%%%%%%%%%%%

\section{Team Member Inclusion and Non-Discrimination Policy}

The NWS Wakefield SKYWARN Amateur Radio Support Team has flourished by embracing the unique background and abilities of every team member. Each individual brings a distinct skill set to the team and we must maintain an environment of inclusion and non-discrimination for all volunteers. We welcome all participants who, with training and reasonable accommodation, can perform the basic functions required of a team member in their position, and we will seek out modified or alternative roles for other members where possible.

We do not tolerate abusive or discriminatory behavior based in whole or in part, on the person's race, color, national origin, age, religion, disability status, gender, sexual orientation, gender identity, veteran/military status, genetic information, or marital status. While our resources are limited, we make every effort to extend reasonable accommodations to volunteers who may have special requirements during the training process and in the routine performance of SKYWARN functions. These accommodations may include facility access considerations; transportation, seating, or dietary arrangements; accessibility devices; and modification of operating procedures as needed.

%%%%%%%%%%%%%%%%%%%%%%%%%%%%%%%%%%%%%%%%%%%%%%%%%%%%%%%%%%%%%%%%%%%%%%%%

\section{Insurance}

The Wakefield SKYWARN Amateur Radio Support Team does not maintain any property or liability insurance on its assets, members, guests, or those of the National Weather Service, facilities used in the course of various training events and other operations, or participants in such training events or operations. 

All participants would be well served by maintaining all necessary insurance on themselves, their equipment, their automobiles, and any such property, and the individual participant, along with any such insurance provided by the participant, shall be responsible for covering any losses incurred while serving the SKYWARN Amateur Radio Support Team in any capacity, including during travel to and from training events, the National Weather Service Forecast Office, meetings, and any other Team business. 

Persons operating at the National Weather Service Forecast Office may have some protections or coverage, possibly to include property and/or liability insurance, under insurance policies in place through the Department of Commerce and the National Weather Service.  Any such coverage, if it exists, is outside the scope of this manual.  Questions on this matter should be directed to the SKYWARN Program Manager or Meteorologist-in-Charge.

%%%%%%%%%%%%%%%%%%%%%%%%%%%%%%%%%%%%%%%%%%%%%%%%%%%%%%%%%%%%%%%%%%%%%%%%

\section{Media Contact Policy}

If any SKYWARN volunteer is contacted by the media, whether or not the contact is related to a SKYWARN activation, they are to direct the media representative to contact the Amateur Radio Coordinator. 

This process ensures that no sensitive, confidential, or otherwise restricted information ends up being ``leaked'' to the media and ensures a positive public image is conveyed through media contacts.  Additionally refer to the section in this manual on ``\nameref{handling-sensitive}'' on page \pageref{handling-sensitive}.

Prior to distributing any materials such as statements, announcements, or press releases to the media or other entities, Area Managers and other SKYWARN volunteers are to consult with the Amateur Radio Coordinator for guidance and approval.

Under no circumstances will SKYWARN members' personal information be given out without their permission, nor will log details or other SKYWARN records be released to any outside party without authorization of the Amateur Radio Coordinator.

%%%%%%%%%%%%%%%%%%%%%%%%%%%%%%%%%%%%%%%%%%%%%%%%%%%%%%%%%%%%%%%%%%%%%%%%

\section{Public Information Officer}

For purposes of ICS Compliance, the Amateur Radio Coordinator shall serve as the team's Public Information Officer (PIO).  This role may be delegated to another team member as may be needed from time to time.

%%%%%%%%%%%%%%%%%%%%%%%%%%%%%%%%%%%%%%%%%%%%%%%%%%%%%%%%%%%%%%%%%%%%%%%%

\section{Social Media Policy}

The team may from time to time utilize one or more social networks or other online tools to collaborate with SKYWARN partners and the general public.  These online resources may be used for any of the following purposes:

\begin{enumerate}
\item To share information about current or projected SKYWARN activations.
\item To disseminate information about upcoming SKYWARN training opportunities.
\item To distribute information of general interest to the SKYWARN community.
\item To distribute or redistribute information about our partner organizations, such as ham radio clubs and emergency service organizations.
\item To provide links to information on severe weather events.
\end{enumerate}

The nature of these social networking tools allows interaction from any number of participants.  Because of the public exposure provided by these services, great care must be taken to maintain the quality and professionalism of posts made by the team and its representatives, while also ensuring that discussions posted by others are relevant and appropriate.

Regarding official posts made by team representatives on social media outlets:

\begin{enumerate}
\item Posts shall be checked before posting to ensure they are composed with proper spelling and grammar.
\item No copyrighted material may be posted without prior authorization and appropriate citation.
\item No posts of a political or religious nature are permitted.
\end{enumerate}

Regarding posts by others:

\begin{enumerate}
\item Posts which contain foul or abusive language shall be edited or deleted.
\item Posts of a political or religious nature are not permitted and are subject to edits or deletion.
\item No copyrighted material may be posted without prior authorization and appropriate citation.
\item Posts which are critical of the team or the National Weather Service are specifically permitted, however, they should be free of foul or abusive language as set forth in this policy, and great care should be exercised in responding to these types of messages.
\end{enumerate}

The Amateur Radio Coordinator remains primarily responsible for social media communications but may delegate all or part of this responsibility to other team members as appropriate.

%%%%%%%%%%%%%%%%%%%%%%%%%%%%%%%%%%%%%%%%%%%%%%%%%%%%%%%%%%%%%%%%%%%%%%%%

\section{Team Member Identification Cards}

Official team photo identification cards are available for purchase by all active team members.  Only orange and green color coded ID badges are valid for access to the Forecast Office and the SKYWARN Radio Desk.  ID badges are color coded as follows:

\textbf{Orange:}  Amateur Radio Coordinator, Area Manager, Assistant Area Manager, and Responder.

\textbf{Blue:}  Net Control Operator

\textbf{Green:}  Support Personnel (VE Team, Tech Team)

\textbf{Black:}  Spotter

Team members are responsible for the purchasing cost of their photo ID badges, including any cost associated with upgrading or downgrading the badge type.  Badges remain the property of the NWS Wakefield SKYWARN Amateur Radio Support Team, with the exception of Spotter badges, which are property of the purchaser.

Net Control Operators, Support Personnel, and Spotters are not required to have ID badges.

Official photo ID badges or an official Visitor Badge must be displayed at all times while on WFO premises.

%%%%%%%%%%%%%%%%%%%%%%%%%%%%%%%%%%%%%%%%%%%%%%%%%%%%%%%%%%%%%%%%%%%%%%%%
%%%%%%%%%%%%%%%%%%%%%%%%%%%%%%%%%%%%%%%%%%%%%%%%%%%%%%%%%%%%%%%%%%%%%%%%
%%%%%%%%%%%%%%%%%%%%%%%%%%%%%%%%%%%%%%%%%%%%%%%%%%%%%%%%%%%%%%%%%%%%%%%%
%%%%%%%%%%%%%%%%%%%%%%%%%%%%%%%%%%%%%%%%%%%%%%%%%%%%%%%%%%%%%%%%%%%%%%%%
%%%%%%%%%%%%%%%%%%%%%%%%%%%%%%%%%%%%%%%%%%%%%%%%%%%%%%%%%%%%%%%%%%%%%%%%
%%%%%%%%%%%%%%%%%%%%%%%%%%%%%%%%%%%%%%%%%%%%%%%%%%%%%%%%%%%%%%%%%%%%%%%%

\chapter{Reporting Criteria}\label{reporting-criteria}

The criteria for reportable events –-- that is, those reports which should be relayed to the NWS –-- are set by the Wakefield WFO and may vary from time to time.

Reporting criteria for convective and tropical events are as follows:
\begin{itemize}
\item Hail pea size or larger.
\item Winds 50 mph or higher.
\item Trees, branches, or power lines down.
\item Any structural damage from wind or hail.
\item Tornadoes and funnel clouds.
\item Rotating wall clouds.
\item Waterspouts.
\item Flooding of ditches, streets, streams, and rivers.
\item Coastal flooding from high tides.
\item Measured rainfall 1” or greater.
\item Fires caused by lightning.
\item Deaths or injuries caused by the storm.
\end{itemize}

Reporting criteria for non-convective winter weather events are as follows:
\begin{itemize}
\item Measured snowfall 1'' or deeper or falling at a rate of 1'' per hour or more.
\item Any ice accretion on roads, trees, etc.
\item Winds 35 mph or higher with ice accretion or significant snow buildup on trees.
\item Trees, branches, or power lines down.
\item Deaths or injuries caused by the storm.
\end{itemize}



\end{document}